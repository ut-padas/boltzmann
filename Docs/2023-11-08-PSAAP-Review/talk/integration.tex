\documentclass[mathserif, aspectratio=169]{beamer}
\usetheme{odenpecos}
\setbeamertemplate{itemize/enumerate body begin}{\fontsize{8.8}{9}\selectfont}
\setbeamertemplate{itemize/enumerate subbody begin}{\fontsize{7.5}{8}\selectfont}
\setbeamertemplate{itemize/enumerate subsubbody begin}{\fontsize{7.5}{8}\selectfont}

% default search path for figures
\graphicspath{{./../fig/}}

\newcommand{\zapspace}{\topsep=0pt\partopsep=0pt\itemsep=0pt\parskip=0pt}

\usepackage{multicol}
\usepackage{multirow}
\usepackage{pict2e}
%\usepackage{esdiff}
\usepackage{multimedia}
\usepackage{verbatim}
\usepackage{mhchem}
\usepackage{tikz}
\usetikzlibrary{arrows}
\usepackage[percent]{overpic}
\usepackage[absolute,overlay]{textpos}
\usepackage{tikz} % Required for flow chart
\usepackage[caption=false]{subfig}
\usepackage{pgfplots}

\newcommand{\overbar}[1]{\mkern 1.5mu\overline{\mkern-1.5mu#1\mkern-1.5mu}\mkern 1.5mu}
\newcommand{\pp}[2]{\frac{\partial #1}{\partial #2}}
\newcommand{\dd}[2]{\frac{d #1}{d #2}}
\newcommand{\DD}[2]{\frac{D #1}{D #2}}
\newcommand{\mm}{\mathbf{minmod}}
\def\etal{{\it et al~}}
\newcommand{\be}{\begin{eqnarray}}
	\newcommand{\ee}{\end{eqnarray}}
\newcommand{\mbb}[1]{\mathbb{#1}} % math blackboard bold
\newcommand{\mcal}[1]{\mathcal{#1}} % math blackboard bold
\newcommand{\mbf}[1]{\mathbf{#1}} % math bold face (for vectors)
\newcommand{\sbf}[1]{\boldsymbol{#1}} % bold face for symbols
\newcommand{\jump}[1]{\llbracket #1 \rrbracket} % jump operator
\newcommand{\avg}[1]{\langle #1 \rangle} % average operator
\newcommand{\rarrow}{\rightarrow}
\newcommand{\Rarrow}{\Rightarrow}
\newcommand{\LRarrow}{\Leftrightarrow}
\newcommand{\vvvert}{|\kern-1pt|\kern-1pt|}
\newcommand{\enorm}[1]{\vvvert #1 \vvvert}
\newcommand{\nutil}{\tilde{\nu}}
\newcommand{\Var}{\mathrm{Var}}
\newcommand{\Cov}{\mathrm{Cov}}


\definecolor{MyDarkGreen}{rgb}{0,0.45,0.08}
\newcommand{\myred}[1]{{\color{red} #1}}
\newcommand{\myblue}[1]{{\color{blue} #1}}
\newcommand{\mygreen}[1]{{\color{MyDarkGreen} #1}}

\newcommand{\sa}{\nu_{\mathrm{sa}}}
\newcommand{\tep}{\tilde{\epsilon}}
\newcommand{\Ssd}{\mathcal{S}} % source term due to slow derivative
\newcommand{\ud}{\,\mathrm{d}}

\newcommand{\Mach}[1]{\ensuremath{\mbox{Ma}_{#1}}}
\newcommand{\Reynolds}{\ensuremath{\mathit{Re}}}
\newcommand{\DensityRat}{\ensuremath{\mathit{DR}}}
\newcommand{\BlowRat}{\ensuremath{\mbox{BR}}}
\newcommand{\VelRat}{\ensuremath{\mathit{VR}}}
\newcommand{\Tau}{\ensuremath{\mathrm{T}}}

\newcommand{\wall}     {\ensuremath{\mathrm{w}}}   % wall subindex
\newcommand{\awall}    {\ensuremath{\mathrm{aw}}}  % adiabatic wall subindex

\newcommand{\commentout}[1]{}

\newcommand{\vect}[1]{\boldsymbol{#1}}
\usepackage{mleftright}
\newcommand{\of}[1]{\mleft( #1 \mright)}
\newcommand{\vth}{v_{\textrm{th}}}
\newcommand{\reals}{\mathbb{R}}
\newcommand{\myint}{\int\limits}
\newcommand{\ddt}[1]{\partial_t #1}
\newcommand{\RR}{\mathbb{R}}
\newcommand{\vr}{v}
\newcommand{\diff}[1]{\, d#1}
\newcommand{\norm}[1]{\left\lVert#1\right\rVert}
%\newcommand{\vtheta}{\theta_{\vect{v}}}
%\newcommand{\vphi}{\varphi_{\vect{v}}}
%\newcommand{\vr}{v_{r}}
\newcommand{\vtheta}{{v_{\theta}}}
\newcommand{\vphi}{v_{\varphi}}
\newcommand{\vomega}{v_{\omega}}
\newcommand{\vrunit}{\hat{\vect{v}}_{r}}
\newcommand{\vthetaunit}{\hat{\vect{v}}_{\theta}}
\newcommand{\vphiunit}{\hat{\vect{v}}_{\varphi}}
\DeclareMathOperator{\variance}{Var}

\usepackage{mathtools}
\DeclarePairedDelimiter\ceil{\lceil}{\rceil}
\DeclarePairedDelimiter\floor{\lfloor}{\rfloor}
\usepackage{tikz}

\usepackage{tcolorbox}
\tcbuselibrary{minted,breakable,xparse,skins}
\definecolor{bg}{gray}{0.95}
\DeclareTCBListing{mintedbox}{O{}m!O{}}{%
	breakable=true,
	listing engine=minted,
	listing only,
	minted language=#2,
	minted style=default,
	minted options={%
		linenos,
		gobble=0,
		breaklines=true,
		breakafter=,
		fontsize=\small,
		numbersep=8pt,
		#1},
	boxsep=0pt,
	left skip=0pt,
	right skip=0pt,
	left=25pt,
	right=0pt,
	top=3pt,
	bottom=3pt,
	arc=5pt,
	leftrule=0pt,
	rightrule=0pt,
	bottomrule=2pt,
	toprule=2pt,
	colback=bg,
	colframe=orange!70,
	enhanced,
	overlay={%
		\begin{tcbclipinterior}
			\fill[orange!20!white] (frame.south west) rectangle ([xshift=20pt]frame.north west);
	\end{tcbclipinterior}},
	#3,
}

\begin{document}
% disable nav
\setbeamertemplate{navigation symbols}{}

% ---------------------------------------------------------------
% Oden/Pecos title page

\hoffset=.16in

\begin{frame}[plain,t]{}
\makeatletter
%\vspace*{0.85cm}
%\vspace*{0.65cm}
\includegraphics[height=0.9in,trim=50 40 40 0, clip]{PMSc_159_university_formal_horizontal.pdf} \newline
%\vspace*{0.3cm}
\begin{columns}[T,onlytextwidth]
\column{.8\textwidth}
{\bf \color{burntorange} \fontfamily{bch}\selectfont 
% -- Set talk title here
Integration efforts on the electron Boltzmann solver with the torch plasma simulator
%Electron Boltzmann solver integration with the torch plasma simulator
% --
}
\end{columns}
\vspace*{.15cm}
\rule{.8\textwidth}{0.6pt} \newline

\vspace*{0.05cm}
\setstretch{0.65}
{\fontfamily{phv}\selectfont
  { \scriptsize
    % -- define presenter, authors here
    \textbf{Milinda Fernando}, Kinetic solvers, Parla, and TPS team members \\
    % --
  }
  {\color{burntorange} \tiny
    % -- define role, meeting event, location, etc
    PSAAP III Annual Review $\cdot$ November 08-09, 2023
    %PSAAP TST Meeting $\cdot$ April 24-25, 2023
    % --
  }
}

\vspace*{1cm}
%\includegraphics[height=0.3in]{figures/pecos_orange1.png}
\begin{columns}
\begin{column}{0.8\linewidth}
\includegraphics[height=0.5in]{oden_pecos_2020_wordmark.png}\\
{\scriptsize \url{https://pecos.oden.utexas.edu}}
\end{column}

\begin{column}{0.2\linewidth}
\includegraphics[height=0.6in]{psaap3-logo.png}
\end{column}
\end{columns}

\end{frame}
\hoffset=0in
% -- end title slide ---------------------------------------------

%\begin{frame}
%	\frametitle{Outline}
%	\begin{itemize}
%		\item Spatially homogeneous Boltzmann equation, $\partial_t f - \frac{\vect{E} q}{m} \cdot \nabla_{\vect{v }}f = C(f)$
%		\item Representation of $f$ (i.e., isotropic + anisotropic correction terms), use spherical harmonics for angular directions + experimentation of basis functions in radial direction. 
%		\begin{itemize}
%			\item Global approximations with Maxwell and Laguerre polynomials.
%			\item Local approximations with linear and higher order B-splines
%		\end{itemize}
%		\item Collision operator (5d integral form)
%		\item Simplifications for the collision operator with analytical integration of angular directions (1d integral form). 
%		\item Equations for the steady-state solution (spatially homogeneous case) 
%		\item Two-term formulation vs. EEDF formulation with diffusion term. 
%		\item Verification with Bolsig+ code.  (Both approaches, importance of the diffusion term), PS 2.4
%		\item Formulation for 1D-space+3D-velocity space Boltzmann equations with some preliminary results for 1d glow discharge problem. 
%		\item Discuss on single GPU implementation with CuPy, Challenges in 1D+3V formulation (i.e., boundary conditions) and Future work, ES 2.6
%	\end{itemize}
%\end{frame}

\begin{frame}
	\frametitle{Where do we fit ?}
	\begin{figure}
		\centering
		\includegraphics[width=0.9\textwidth]{where_we_fit.png}
	\end{figure}
\end{frame}

%\begin{frame}
%	\frametitle{Outline}
%\end{frame}

\begin{frame}
	\frametitle{Torch plasma simulator}
	\begin{columns}
		\begin{column}{0.48\textwidth}
			\textbf{TPS}
			%\vspace{0.25in}
			\footnotesize
			\begin{align*}
				&\partial_t n_i + \nabla_{\vect{x}} \cdot \vect{J_{n_i}}  = k_i n_0 n_i  \color{gray}{\text{ in } \Omega_x \times (0,T]}\\
				%\partial_t n_0 + \nabla_{\vect{x}} \cdot \vect{J_{n_0}}  = -k_i n_0 n_i \text{ in } \Omega_x \times (0,T]\\
				&\text{conservation of mass, momentum} \\
				&\quad  \text{and energy for } \vect{u}, n_0, T_g \\
				%\text{conservation of momentum for }  r^{+}, Ar  \\
				&\text{Maxwell's equations for } \vect{E}
			\end{align*}
			\begin{itemize}
				\item $T_g=T_e$ %Assumes local thermodynamic equilibrium, (i.e., $T_e = T_g$) and quasi-neutrality (i.e., $n_e=n_i$)
				\item Quasi-neutrality: $n_e = n_i$ 
				\item $k_i \approx$ Maxwellian EEDF at $T_g$
			\end{itemize}
		\end{column}
		\begin{column}{0.48\textwidth}
			\textbf{TPS + Boltzmann}
			%\vspace{0.25in}
			\footnotesize
			\begin{align*}
				&\partial_t n_i + \nabla_{\vect{x}} \cdot \vect{J_{n_i}}  = k_i n_0 n_i  \color{gray}{\text{ in } \Omega_x \times (0,T]}\\
				%\partial_t n_0 + \nabla_{\vect{x}} \cdot \vect{J_{n_0}}  = -k_i n_0 n_i \text{ in } \Omega_x \times (0,T]\\
				&\text{conservation of mass, momentum} \\
				&\quad  \text{and energy for } \vect{u}, n_0, T_g \\
				%\text{conservation of momentum for }  r^{+}, Ar  \\
				&\text{Maxwell's equations for } \vect{E}\\
				&\partial_t f  + \vect{v} \cdot \nabla_{\vect{x}} f -\frac{\vect{E} q}{m} \cdot \nabla_{\vect{v }}f = C(f, n_0, n_i, T_g) \color{gray}{\text{ in } \Omega_x \times \Omega_v \times (0,T]}
			\end{align*}
			\begin{itemize}
				\item $T_e \sim \int_{\vect{v}} \norm{\vec{v}}^2 \hat{f} \diff{\vect{v}}$
				\item Quasi-neutrality: $n_e = n_i$ 
				\item $k_i = \int_{\vect{v}} \sigma(\norm{\vect{v}}) \hat{f} \diff{\vect{v}}$ where $\hat{f} = \frac{f}{\int_{\vect{v}} f \diff{\vect{v}}}$
				%\item 7D problem
			\end{itemize}
		\end{column}
	\end{columns}

\end{frame}

\begin{frame}
	\frametitle{TPS + Boltzmann}
	% \begin{itemize}
	% 	\item Importance of spatially coupled Boltzmann on torch QoIs to be determined
	% 	\item Current approach: Batch of spatially homogeneous Boltzmann solves for the $\Omega_x$
	% \end{itemize}
	\vspace{0.2in}
	\begin{columns}
		\begin{column}{0.48\textwidth}
			\textbf{Two-way coupling}\\
			\begin{figure}
				\resizebox{0.9\textwidth}{!}{
					\centering
					\begin{tikzpicture}[block/.style={draw=black,thick, inner sep=2pt, rounded corners, minimum width=2cm, minimum height=1.2cm, fill=rightfooterorange,font={\small}}, shift=(current page.center)]
						\begin{scope}[yshift=10cm,xshift=5cm]
							% \draw[lightgray]
							% (current page.north) -- (current page.south)
							% (current page.west)  -- (current page.east);
							\node[block] (A) at (-9, 3) {TPS};
							\node[block] (C) at (-3 , 3.001) {Boltzmann};
							% \node[block] (D) at (0.002 , 0) {electron kinetic coefficients ($T_g, E/n_0, n_e/n_0, E$)};
							% \draw [->,very thick] (B) -- node [text width=3cm, midway, above, align=center,xshift =-0.8cm] {with Maxwellian assumption $f = M_{T_e}(\vect{v})$} (A);\pause
							% \draw [->,very thick] (B) -- node [text width=3cm, midway, above, align=center,xshift = 1.6cm, yshift=-0.5cm] {collision operator assembly} (C);
							% \draw [->,very thick] (C) -- node [text width=3cm, midway, above, align=center, yshift=-0.5cm, xshift=1.2cm] {static E-field assumption} (D);
							% \draw [->,very thick] (D) -- node [text width=3cm, midway, above, align=center,xshift =-1.2cm, yshift=-1.8cm] {$T_e$-based \\ interpolation for rates, kinetic coefficients} (A);\pause
							\draw [->,very thick] (A) to[bend left  = 8] node [text width=3cm, midway, above, align=center,yshift=0cm] {$T_g, \vect{E}, n_i, n_0$} (C);
							\draw [->,very thick] (C) to[bend left  = 8] node [text width=3cm, midway, below, align=center,xshift=0.0cm] {$k_i$,  $T_e$} (A);
						\end{scope}
				\end{tikzpicture}}
			\end{figure}
		\end{column}
		\begin{column}{0.48\textwidth}
			\textbf{One-way coupling}\\
			\begin{figure}
				\resizebox{0.9\textwidth}{!}{
					\centering
					\begin{tikzpicture}[block/.style={draw=black,thick, inner sep=2pt, rounded corners, minimum width=2cm, minimum height=1.2cm, fill=rightfooterorange,font={\small}}, shift=(current page.center)]
						\begin{scope}[yshift=10cm,xshift=5cm]
							% \draw[lightgray]
							% (current page.north) -- (current page.south)
							% (current page.west)  -- (current page.east);
							\node[block] (A) at (-9, 3) {TPS};
							\node[block] (C) at (-3 , 3.001) {Boltzmann};
							\node[] (B) at (-9, 1.2) {$k_i^{tps}$};
							\node[] (D) at (-3, 1.2) {$k_i^{bte}$};

							% \node[block] (D) at (0.002 , 0) {electron kinetic coefficients ($T_g, E/n_0, n_e/n_0, E$)};
							% \draw [->,very thick] (B) -- node [text width=3cm, midway, above, align=center,xshift =-0.8cm] {with Maxwellian assumption $f = M_{T_e}(\vect{v})$} (A);\pause
							% \draw [->,very thick] (B) -- node [text width=3cm, midway, above, align=center,xshift = 1.6cm, yshift=-0.5cm] {collision operator assembly} (C);
							% \draw [->,very thick] (C) -- node [text width=3cm, midway, above, align=center, yshift=-0.5cm, xshift=1.2cm] {static E-field assumption} (D);
							% \draw [->,very thick] (D) -- node [text width=3cm, midway, above, align=center,xshift =-1.2cm, yshift=-1.8cm] {$T_e$-based \\ interpolation for rates, kinetic coefficients} (A);\pause
							\draw [->,very thick] (A) to[bend left  = 8] node [text width=3cm, midway, above, align=center,yshift=0cm] {$T_g, \vect{E}, n_i, n_0$} (C);

							\draw [->,very thick] (A) to node [text width=3cm, midway, above, align=center,yshift=0cm] {} (B);
							\draw [->,very thick] (C) to node [text width=3cm, midway, above, align=center,yshift=0cm] {} (D);
							\draw [<->,very thick] (B) to node [text width=3cm, midway, below, align=center,xshift=0.0cm] {compare} (D);
						\end{scope}
				\end{tikzpicture}}
			\end{figure}
		\end{column}
	\end{columns}
	\begin{itemize}
		\item Completed 1-way integration with 0D-space BTE
	\end{itemize}
	% Current integration
	% \begin{itemize}
	% 	\item Focus on weak coupling to evaluate modeling errors
	% 	\item Take, $n_0, T_g, n_i, \vect{E}$ from TPS code, and compare $k_i\approx$ Maxwellian EEDF to more generic electron model (i.e., Boltzmann equation) 
	% \end{itemize}
\end{frame}

\begin{frame}
	\frametitle{Batched 0D-BTE}
	%\vspace{-0.1in}
	\begin{itemize}
		\item Launch 0D-space BTE solve for each TPS spatial point ($\approx$ 256 DoFs per grid point)\\
		% \item $N_{v} \approx$ 256 DoFs per grid point
		% \item $\vect{v}$-space truncation, determining EEDF energy range
		% \item Supports steady-state and transient solutions
		\textbf{\textcolor{orange}{Amortize BTE v-space grid setup} }
	\begin{center}
		\includegraphics[width=0.6\textwidth]{tps_0d_clusters.png}
		% \resizebox{0.48\textwidth}{!}{
		% \begin{tikzpicture}
		% 	\begin{scope}[xshift=0cm]
		% 		\draw[thick, blue!80!black!80] (0,0) -- (4,0);
		% 		\draw[thick, blue!80!black!80] (4,-0.2) --node[below, yshift=-0.1cm]{$\varepsilon^{max}_{0}$} (4,0.2);
		% 		\draw[thick, blue!80!black!80] (0,-0.2) --node[below, yshift=-0.1cm]{$0$} (0,0.2);	
		% 		\node at (2,-0.5) {cluster 0};
		% 	\end{scope}
			
		% 	\begin{scope}[xshift=5cm]
		% 		\draw[thick, green!80!black!80] (0,0) -- (4,0);
		% 		\draw[thick, green!80!black!80] (4,-0.2) --node[below, yshift=-0.1cm]{$\varepsilon^{max}_{1}$} (4,0.2);
		% 		\draw[thick, green!80!black!80] (0,-0.2) --node[below, yshift=-0.1cm]{$0$} (0,0.2);	
		% 		\node at (2,-0.5) {cluster 1};
		% 	\end{scope}

		% 	% \begin{scope}[xshift=10cm]
		% 	% 	\draw[thick, black] (0,0) -- (4,0);
		% 	% 	\draw[thick, black] (4,-0.2) --node[below, yshift=-0.1cm]{$\varepsilon^{max}_{2}$} (4,0.2);
		% 	% 	\draw[thick, black] (0,-0.2) --node[below, yshift=-0.1cm]{$0$} (0,0.2);	
		% 	% 	\node at (2,-0.5) {cluster 2};
		% 	% \end{scope}

		% 	% \begin{scope}[xshift=15cm]
		% 	% 	\draw[thick, black] (0,0) -- (4,0);
		% 	% 	\draw[thick, black] (4,-0.2) --node[below, yshift=-0.1cm]{$\varepsilon^{max}_{3}$} (4,0.2);
		% 	% 	\draw[thick, black] (0,-0.2) --node[below, yshift=-0.1cm]{$0$} (0,0.2);	
		% 	% 	\node at (2,-0.5) {cluster 3};
		% 	% \end{scope}
		% \end{tikzpicture}}
	\end{center}
	\begin{itemize}
		\item K-means clustering based on gas temperature $T_g(\vect{x})$. 
		\item $\vect{v}$-space grid truncation $\quad\Rightarrow\quad$ $\overbar{T}_g$ $\quad\Rightarrow\quad$ set of $C_{en}, C_{ee}, A_v$ for cluster $i$%For each cluster EEDF is resolved up to $\varepsilon(\bar{T}_g)$
	\end{itemize}
	\pause
	% \begin{itemize}
	% 	\item Right-hand-side computation done by stacking BTE DoFs 
	% \end{itemize}
	{\footnotesize
	\begin{align*}
		\underbrace{C_{en}, C_{ee}, A_{v}}_{\text{pre-computed v-space operator for $i^{\text{th}}$-cluster}} \underbrace{\begin{bmatrix}
			%\vdots & \vdots & \hdots &\vdots\\
			f_1    & f_2    & \hdots & f_{N_A} \\
			\vdots & \vdots & \hdots &\vdots
		\end{bmatrix}}_{\text{batched DoFs for $i^{\text{th}}$-cluster}}
	\end{align*}}
	\end{itemize}
\end{frame}

% \begin{frame}
% 	\frametitle{Batched 0D-BTE}
% 	\textbf{\textcolor{orange}{Amortize BTE v-space grid setup} }
% 	\begin{center}
% 		\includegraphics[width=0.6\textwidth]{tps_0d_clusters.png}
% 		% \resizebox{0.48\textwidth}{!}{
% 		% \begin{tikzpicture}
% 		% 	\begin{scope}[xshift=0cm]
% 		% 		\draw[thick, blue!80!black!80] (0,0) -- (4,0);
% 		% 		\draw[thick, blue!80!black!80] (4,-0.2) --node[below, yshift=-0.1cm]{$\varepsilon^{max}_{0}$} (4,0.2);
% 		% 		\draw[thick, blue!80!black!80] (0,-0.2) --node[below, yshift=-0.1cm]{$0$} (0,0.2);	
% 		% 		\node at (2,-0.5) {cluster 0};
% 		% 	\end{scope}
			
% 		% 	\begin{scope}[xshift=5cm]
% 		% 		\draw[thick, green!80!black!80] (0,0) -- (4,0);
% 		% 		\draw[thick, green!80!black!80] (4,-0.2) --node[below, yshift=-0.1cm]{$\varepsilon^{max}_{1}$} (4,0.2);
% 		% 		\draw[thick, green!80!black!80] (0,-0.2) --node[below, yshift=-0.1cm]{$0$} (0,0.2);	
% 		% 		\node at (2,-0.5) {cluster 1};
% 		% 	\end{scope}

% 		% 	% \begin{scope}[xshift=10cm]
% 		% 	% 	\draw[thick, black] (0,0) -- (4,0);
% 		% 	% 	\draw[thick, black] (4,-0.2) --node[below, yshift=-0.1cm]{$\varepsilon^{max}_{2}$} (4,0.2);
% 		% 	% 	\draw[thick, black] (0,-0.2) --node[below, yshift=-0.1cm]{$0$} (0,0.2);	
% 		% 	% 	\node at (2,-0.5) {cluster 2};
% 		% 	% \end{scope}

% 		% 	% \begin{scope}[xshift=15cm]
% 		% 	% 	\draw[thick, black] (0,0) -- (4,0);
% 		% 	% 	\draw[thick, black] (4,-0.2) --node[below, yshift=-0.1cm]{$\varepsilon^{max}_{3}$} (4,0.2);
% 		% 	% 	\draw[thick, black] (0,-0.2) --node[below, yshift=-0.1cm]{$0$} (0,0.2);	
% 		% 	% 	\node at (2,-0.5) {cluster 3};
% 		% 	% \end{scope}
% 		% \end{tikzpicture}}
% 	\end{center}
% 	\begin{itemize}
% 		\item K-means clustering based on gas temperature $T_g(\vect{x})$. 
% 		\item $\vect{v}$-space grid truncation $\quad\Rightarrow\quad$ $\overbar{T}_g$ $\quad\Rightarrow\quad$ set of $C_{en}, C_{ee}, A_v$ for cluster $i$%For each cluster EEDF is resolved up to $\varepsilon(\bar{T}_g)$
% 	\end{itemize}
% 	\pause
% 	% \begin{itemize}
% 	% 	\item Right-hand-side computation done by stacking BTE DoFs 
% 	% \end{itemize}
% 	{\footnotesize
% 	\begin{align*}
% 		\underbrace{C_{en}, C_{ee}, A_{v}}_{\text{pre-computed v-space operator for $i^{\text{th}}$-cluster}} \underbrace{\begin{bmatrix}
% 			%\vdots & \vdots & \hdots &\vdots\\
% 			f_1    & f_2    & \hdots & f_{N_A} \\
% 			\vdots & \vdots & \hdots &\vdots
% 		\end{bmatrix}}_{\text{batched DoFs for $i^{\text{th}}$-cluster}}
% 	\end{align*}}
% 		%\item \texttt{Numpy} and \text{Cupy} is used for linear algebra computations
% 		%\item \textbf{Parla} is used for parallelization between clusters
	
% 	% Boltzmann + Parla + MPI
% 	% \begin{itemize}
% 	% 	\item \textbf{Parla} is used for parallelization between clusters
% 	% 	\item \textbf{MPI} is used for distributed memory parallelism
% 	% 	\item 0D-BTE batched solver is integrated with TPS code
% 	% \end{itemize}
% \end{frame}

\begin{frame}
	\frametitle{Comparison between different models}
	\begin{itemize}
		\item \textbf{Maxwellian} : $T_e=T_g$ and $k_i^{A} = \int_{\vect{v}} \sigma \textcolor{blue}{f^{\text{Maxwellian}}} $  (currently used in TPS code)%from Maxwellian EEDF %and $f_M(\vect{v}, t)=M_{T_e}(\vect{v})$ and $k_i^{M} = \mathcal{K}f_M$
		\item \textbf{Static E-field} : Batched \textcolor{orange}{steady-state} 0D-Boltzmann, $k_i^{B} = \int_{\vect{v}} \sigma \textcolor{blue}{f^{B}} $ 
		\item \textbf{Oscillatory E-field} : Batched \textcolor{orange}{transient} 0D-Boltzmann, $k_i^{C} = \frac{1}{T} \int_{T} \int_{\vect{v}} \sigma \textcolor{blue}{f^{C}} $ %Batched 0D-Boltzmann solver with oscillatory $\vect{E}$ field with cycle averaged rates, $k_i^{C} = \mathcal{K}f$ with weak-coupling
		%\item In-terms of accuracy, $\text{\textbf{A}} < \text{\textbf{B}} < \text{\textbf{C}} < \text{Fully coupled TPS + Boltzmann}$
		\vspace{0.25in}
		\item Reaction rate errors, for example, $\cfrac{|k_i^{A} n_0 n_i  - k_i^{B} n_0 n_i|}{\norm{k_i^{A} n_0 n_i}_{\infty}}$ %> \epsilon = 10^{-2}$
	\end{itemize}
	%We compare \textbf{B} vs. \textbf{A}, \textbf{C} vs. \textbf{A} and \textbf{B} vs. \textbf{C}
\end{frame}

\begin{frame}
	\frametitle{TPS + 0D-Boltzmann}
	\vspace{-0.4in}
	\begin{figure}
		\centering
		\includegraphics[width=0.48\textwidth]{te_vs_tg_tps_bte.png}
		\includegraphics[width=0.48\textwidth]{rates_tps_bte.png}
	\end{figure}
	\begin{itemize}
		\item Static E-field leads higher $T_e$ and $k_i$ compared to oscillatory E-field with cycle averaged QoIs. 
	\end{itemize}
\end{frame}

% \begin{frame}
% 	\frametitle{Maxwellian vs. steady-state 0D-Boltzmann}
% 	\begin{figure}
% 		\begin{center}
% 			\includegraphics[width=1.05\textwidth]{tps_0d2v_ss.png}
% 		\end{center}
% 	\end{figure}
% 	\textbullet~ Relative errors for $k_i n_0 n_i$ 50\% compared to Maxwellian EEDF
% \end{frame}

\begin{frame}
	\frametitle{Transient 0D-Boltzmann vs. Maxwellian}
	\begin{figure}
		\begin{center}
			%\hspace{-0.6in}
			\includegraphics[width=1.05\textwidth]{tps_0d2v_ts1.png}
		\end{center}
	\end{figure}
	\textbullet~ Relative errors for $k_i n_0 n_i$ 200\% compared to Maxwellian EEDF
\end{frame}


% \begin{frame}
% 	\frametitle{steady-state Vs. transient 0D-Boltzmann}
% 	% \begin{itemize}
% 	% 	%\item \textbf{A} : Assume LTE $T_e=T_g$ and $f_M(\vect{v}, t)=M_{T_e}(\vect{v})$ and $k_i^{M} = \mathcal{K}f_M$
% 	% 	\item \textbf{B} : Batched 0D-Boltzmann solver with static $\vect{E}$ , $k_i^{B} = \mathcal{K}f$ with weak-coupling
% 	% 	\item \textbf{C} : Batched 0D-Boltzmann solver with oscillatory $\vect{E}$ field with cycle averaged rates, $k_i^{C} = \mathcal{K}f$ with weak-coupling
% 	% \end{itemize}
% 	\begin{figure}
% 		\begin{center}
% 			%\hspace{-2.25in}
% 			%\vspace{-0.5in}
% 			\includegraphics[width=0.8\textwidth]{tps_0d2v_ts_vs_ss.png}
% 		\end{center}
% 	\end{figure}
% \end{frame}


\begin{frame}[fragile]
	\frametitle{TPS + Python + Boltzmann + Parla + MPI}
	\begin{columns}
		\begin{column}{0.48\textwidth}
			\begin{figure}
				\centering
				\includegraphics[width=1.0\columnwidth]{software_integration.png}
			\end{figure}		
		\end{column}
		\begin{column}{0.50\textwidth}
			\begin{mintedbox}{python}%[break at=.8\textheight]
ts_1 = TaskSpace("T")
for idx in range(num_clusters):
@spawn(ts_1[idx], placement=[p1[idx]], dependencies=ts_0[idx], vcus=0.0)
def t1():
  try:
    ff , qoi = bte.solve(idx, ...)
  except:
    print("solver failed")
			\end{mintedbox}
		\end{column}
	\end{columns}
	\begin{itemize}
		\item Information from Boltzmann to TPS is exchanged with interface class
		% \item Spatial points are clustered using heavy temperature
		% \item Different clusters $\rightarrow$ different $\epsilon_{max}$ truncation
		\item Parallelization between clusters via \textbf{Parla}
	\end{itemize}
\end{frame}

\begin{frame}[fragile]
	\frametitle{2-way coupling: TPS + Boltzmann}
	\begin{itemize}
		\item Main loop is written in Python ochestrating between different codes. 
	\end{itemize}
	\begin{mintedbox}{python}%[break at=.8\textheight]
import libtps
from   bte_0d3v_batched import BoltzmannSolver
comm = MPI.COMM_WORLD
tps = libtps.Tps(comm)
boltzmann = Boltzmann0D2VBactchedSolver(tps, comm)
interface = libtps.Tps2Boltzmann(tps)
while it < max_iters:
	tps.solveStep()
	tps.push(interface)
	boltzmann.fetch(interface)
	boltzmann.solve()
	boltzmann.push(interface)
	tps.fetch(interface)
	it = it+1
	\end{mintedbox}
	\end{frame}


\begin{frame}[fragile]
	\frametitle{Parallel scalability: TPS + Boltzmann}
%	TPS + 0D-Boltzmann problem setup
	\begin{itemize}
		\item TPS spatial points = 7598, DoF per point = 256, Total DoFs $\approx$ 2M, time for 200 implicit timesteps
		%\item runtime for 200 implicit timesteps
		\item Sclability study was performed in TACC's Lonestar 6 %$2\times 128$ AMD EPYC 7763 64-Core Processor with, 3-40GB Nvidia A100 GPU 
	\end{itemize}
	\begin{figure}
		\centering
		\begin{tikzpicture}
			\begin{axis}[
				width=10cm,
				height=5cm,
				ylabel={runtime (s)},
				xlabel={GPUs},
				symbolic x coords={3,6,12},
				xtick=data,
				grid=major
			]
			\addplot[-,mark=o,blue,thick] table[x={gpus}, y ={solve_max} ]{../ls6/tpp_0d2v_ss.dat};
		%\legend{Data1,Data2,Data3}
		\end{axis}
		\end{tikzpicture}
	\end{figure}
	\begin{itemize}
		\item 0D-Boltzmann grid setup $\approx$ 20s (independent of number of compute nodes)
		\item Parallel efficiency 78\%, 82\% $\rightarrow$ better load-balance with increasing partitions
	\end{itemize}
\end{frame}



% \begin{frame}
% 	\frametitle{TPS + 0D2V Boltzmann weak coupling}
% 	\vspace{-0.2in}
% 	\begin{enumerate}
% 		\item Let $\vect{U}_0$ be the initial state for TPS (3-species, axisymmetric case)
% 		\item Evolve TPS solution until time-periodic $\vect{U}_{s}$ solution
% 		\item Extract TPS fields from $\vect{U}_s$ and start BTE solve with $E(\vect{x}, t) = E_x \cos(\omega t) + E_y \sin(\omega t)$, with Maxwellian initial conditions ($\vect{f}_0$)
% 		\item Evolve BTE solution $\vect{f}_0$ until time-periodic solutions $\vect{f}_s$
% 		\item Compute cycle-averaged reaction rates from $\vect{f}_s$ and compare with the rates coefficients used in the TPS code
% 	\end{enumerate}
% \pause
% \textbf{Problem setup}
% \begin{columns}
% 	\begin{column}{0.35\textwidth}

% 		\begin{itemize}
% 			\item TPS mesh elements 7598
% 			\item Number of spatial clusters 4
% 			\item DoFs per spatial point 256, i.e.,  2M DoFs total
% 			\item Number of timesteps 4000
% 			\item Number of GPUs 2 Nvidia A100
% 		\end{itemize}		
% 	\end{column}
% 	\begin{column}{0.7\textwidth}
% 		\begin{itemize}
% 			\item BTE grid setup time $\approx$ 29.3 s
% 			\item Assuming static E-field (direct steady-state solve) $\approx$ 8.07 s
% 			\item Cycle-averaged solution with time-integration (2 cycles, 4000 timesteps) $\approx$ 1314.0 s
% 		\end{itemize}
% 	\end{column}
% \end{columns}
% \end{frame}
% \begin{frame}
% 	\frametitle{TPS + 0D2V Boltzmann weak coupling}
% 	\begin{figure}
% 		\centering
% 		\includegraphics[width=0.5\textwidth]{initial_boltzmann_rxn_comparison.png}
% 	\end{figure}
% \end{frame}

% \begin{frame}
% 	\frametitle{TPS + 0D2V Boltzmann weak coupling}
% 	\vspace{-0.3in}
% 	\begin{figure}
% 		\centering
% 		\includegraphics[width=0.9\textwidth]{tps_bte_comparison1.png}
% 	\end{figure}
% \end{frame}
%\begin{frame}
%	\frametitle{0D-Space Boltzmann one-way coupling}
%	
%\end{frame}

\begin{frame}
	\frametitle{Conclusions \& future work}
	\begin{itemize}
		\item Observe significant errors between Maxwellian EEDF vs. 0D-Boltzmann for electrons %Solver with TPS + Boltzmann + Parla + MPI 
		\item Characterize $k_i$ model error sensitivity for torch QoIs
		\item Detailed performance analysis for Batched Boltzmann
		\item Explore 1D, 2D space Boltzmann for torch 
	\end{itemize}
	% \vspace{0.125in}
	% Ongoing Work
	% \begin{itemize}
	% 	\item Characterize sensitivity for electron kinetic coefficients to torch stability profiles
	% 	\item HPC \& performance for TPS + Boltzmann
	% 	\item 1D2V Boltzmann setup for glow discharge
	% \end{itemize}
	\pause
	\begin{center}
		Questions ? \\
		Thank You!
	\end{center}
\end{frame}
\end{document}
