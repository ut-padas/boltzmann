\begin{frame}[fragile]
	\frametitle{Boltzmann Equation}
%	\raisebox{-0.5\height}{\begin{overpic}[height=0.09\textheight]{pecos_roadmap_tst_1.5.pdf} % figure + highlight box
%			\put(-1,8){\filldraw [semithick, fill=rightfooterorange, fill opacity=0.1, draw=gray, rounded corners] (1.9,1.2) rectangle +(1.15,0.7);}
%	\end{overpic}}
	\begin{itemize}
		\item Why ? Electron distribution function defines the transport and kinetic properties, and its evolution is described by the Boltzmann equation.
		%The Boltzmann equation describes the evolution of the electron distribution function $f=f(\vect{x}, \vect{v}, t)$. %
		\begin{align}
			\partial_t f + \vect{v}\cdot \nabla_{\vect{x}} f  - \frac{\vect{E} q}{m} \cdot \nabla_{\vect{v }}f = C(f)
		\end{align}
		\item Challenges: 6+1 dimensions
		\item First, we consider spatially homogeneous electron Boltzmann equation, focusing on the representation of $f(\vect{v},t)$
		\begin{itemize}
			\item Use spherical harmonics for angular directions. 
			\item Radial direction approximations, with global and local approximations
		\end{itemize}
		\item \textbf{Goal}: Accurate representation of $f$, with minimum DOFs. 
	\end{itemize}
\end{frame}

\begin{frame}
	\frametitle{Steady-state solutions}
	\begin{itemize}
		\item We consider, spatially homogeneous case, with distribution function representation as described above.  
		\item Electric field accelerate electrons (adds energy), while collision causes energy loss.
		\item Therefore, normalized distribution function $\hat{f}\of{\vect{v},t}$ should reach a steady state.  
		$
		\displaystyle
		\partial_t \hat{f} = -(u^T C \hat{f}) \hat{f} + (C+E)\hat{f} \text{ where } \hat{f}(\vect{v},t) = \frac{f(v,t)}{\myint_{R^3} f(\vect{v},t) \diff{\vect{v}}}
		$
		\item We can directly solve the above to compute the steady-state solution.
		$
		\displaystyle
		\quad
		\partial_t (\hat{f}) = 0 \ \ \  \Rightarrow \ \ \ -(u^T C \hat{f}) \hat{f} + (C+E)\hat{f} =0
		$ with $u^T \hat{f}-1=0$
	\end{itemize}
\end{frame}

\begin{frame}
	\frametitle{LXCAT cross section data}
		\centering
		\includegraphics[width=0.7\textwidth]{g0_g2_cs.png}
\end{frame}

%\begin{frame}
%	\frametitle{Investigation of different bases}
%%	\begin{itemize}
%%		\item Choice of basis functions in the radial direction
%%		\begin{itemize}
%%			\item Global approximations with global polynomials
%%			\item Local approximations with B-Splines with local support 
%%		\end{itemize}
%%	\end{itemize}
%	\small
%	\begin{align*}
%		\textrm{Assoc. Laguerre poly:}
%		& \quad \Phi_n\of{v} = L_n\of{v^2}, &&
%		\quad 
%		\myint_{0}^{+\infty} v^2 e^{-v^2} L_n\of{v^2} L_{n^\prime}\of{v^2} \ud v \sim \delta_{nn^\prime}
%		\\
%		\textrm{Maxwell (speed) poly:}
%		& \quad \Phi_n\of{v} = P_n\of{v}, &&
%		\quad 
%		\myint_{0}^{+\infty} v^2 e^{-v^2} P_n\of{v} P_{n^\prime}\of{v} \ud v \sim \delta_{nn^\prime}
%		\\
%		\textrm{B-Splines:}
%		& \quad \Phi_n\of{v} = B_n\of{v}, && 
%		%\quad 
%		%N_n\of{v} = 1 - \frac{|x-x_n|}{\Delta x},\quad x_{n-1} < x < x_{n+1}
%	\end{align*}	
%\end{frame}

\begin{frame}
	\frametitle{Collision operator}
	\begin{itemize}
		\item Encapsulate the physics of underlying collisions. In weak form, 
		$
		\displaystyle
		\quad 
		\myint_{R^3} C \phi\of{\vect{v}_e} \diff{\vect{v}_e} 
		=
		\myint_{R^3} \myint_{R^3} \myint_{S^2} 
		B\of{\vect{v}_e, \vect{v}_0, \vect{\omega}} 
		f_e\of{\vect{v}_e} f_0\of{\vect{v}_0} 
		\left(
		\phi\of{\vect{v}_e^\text{post}\of{\vect{v}_e, \vect{v}_0, \vect{\omega}}} 
		- \phi\of{\vect{v}_e} 
		\right)
		\diff{\vect{v}_0} \diff{\vect{v}_e} \diff{\vect{\omega}}
		$
		\item Assuming $f_0\of{\vect{v}} = n_0 \delta\of{\vect{v}}$, we can simplify, 
		$
		\displaystyle
		\quad 
		\myint_{R^3} C \phi\of{\vect{v}_e} \diff{\vect{v}_e} 
		=
		n_0 \myint_{R^3} \myint_{S^2} 
		B\of{\vect{v}_e, 0, \vect{\omega}} 
		f_e\of{\vect{v}_e}
		\left(
		\phi\of{\vect{v}_e^\text{post}\of{\vect{v}_e, 0, \vect{\omega}}} 
		- \phi\of{\vect{v}_e} 
		\right)
		\diff{\vect{v}_e} \diff{\vect{\omega}}
		$
		\item Collision probability kernel $B\of{\vect{v}, \vect{\omega}}=\norm{\vect{v}} \underbrace{\sigma(\norm{\vect{v}}, \vect{\omega})}_{\text{LXCAT experimental data}}$
		\item Currently we consider, 
		\begin{itemize}
			\item Elastic    : $e + Ar \rightarrow e + Ar$ 
			\item Ionization : $e + Ar \rightarrow 2e + Ar^+$ 
		\end{itemize}
		\item Cross sections might not be continuous, especially for reactions with threshold energy.  
	\end{itemize}
\end{frame}

\begin{frame}
	\frametitle{Simplifications for the collision operator}
	\begin{itemize}
		\item For the moment assume lm=(0,0), (1,0) modes, (i.e., two term expansion of distribution function)
		\item $\vect{v}^{\prime} = (v_r^\prime, v_\theta^\prime, v_\phi^\prime) =\vect{v}^{post}\of{\vect{v},\vect{0},\vect{\omega}}$
		\begin{align*}
			\cos v_\theta^\prime = \cos v_\theta \cos\chi + \sin v_\theta \sin \chi \cos\of{v_\phi-\phi}
		\end{align*}
		\item Using the spherical harmonics addition theorem, we can write. 
		\begin{align*}
			P_l^{0} = P_l\of{\cos v_\theta^\prime} =  P_l\of{\cos v_\theta} P_l\of{\cos v_\phi} + 2\sum_{m=1}^{l} P_l^{m}\of{\cos v_\theta} P_l^{m}\of{\cos v_\phi} \cos \of{ m (v_\phi-\phi)}
		\end{align*}
		\item Analytically integrating out the angles, we can write,
		\begin{align*}
			C^{pq0}_{kl0}  = n_0 \myint_{v} v^3 \sigma\of{v} \phi_k\of{v} \delta_{ql} \of{\psi_p\of{v^{post}}\delta_{q0} - \psi_p\of{v}} \diff{v}  
		\end{align*}
	\end{itemize}
\end{frame}

\begin{frame}
	\frametitle{EEDF formulation}
	\begin{itemize}
		\item Projecting only for the spherical basis, we can write, 
		\begin{center}
		$
		\displaystyle
		\quad
		\frac{d}{dt} f_{0,0} - E 
		\left( \frac{1}{\sqrt{3}} \frac{d}{d\vr} f_{1,0} 
		+  \frac{2}{\sqrt{3}} \frac{1}{\vr} f_{1,0} \right) = \tilde{C}^{00}_{00} f_{0,0} = \tilde{C_0} f_{0,0}
		$
		$\frac{d}{dt} f_{1,0} - E 
		\left( \frac{1}{\sqrt{3}} \frac{d}{d\vr} f_{0,0} \right) =  \tilde{C}^{10}_{10} f_{1,0} = \tilde{C_1} f_{1,0}
		$
		\end{center}
		\item Based on the weak form of the $C^{10}_{10}$, 
		\begin{center}
			$
			\displaystyle
			\quad
			C^{10}_{10} = -n_0 \myint_{R} v^3 \sigma\of{\varepsilon} \phi(v) \psi(v) dv \implies  \tilde{C}^{10}_{10} = -n_0 \varepsilon^{1/2} \gamma \sigma\of{\varepsilon} $
		\end{center} Let $\mu = \cfrac{\dot{n}_e}{n_e}$ be the growth rate due to collisions. In the weak form, 
		\begin{center}
			$\partial_t f(t,v) = (C + E)f \implies \mu = \frac{\dot{n}_e\of{t}}{n_e\of{t}} = \myint_{\vect{V}} C f \diff{\vect{v}} = u^T C \hat{f} \text{ where } \hat{f} = f/n_e$
		\end{center}
	\end{itemize}
\end{frame}

\begin{frame}
	\frametitle{EEDF formulation}
	\begin{itemize}
		\item For steady state we can write, 
		\begin{center}
			$
			\displaystyle
			\quad
			\partial_t \hat{f_1} = \frac{1}{n_e} \partial_t f_1 - \frac{\dot{n_e}}{n_e} \hat{f_1} =0 \implies $
			$
			\displaystyle
			\quad
			\hat{f_1} = \frac{E}{\sqrt{3}} \frac{\partial_{v}\hat{f_0}}{(n_0 \varepsilon^{1/2}\gamma \sigma\of{\varepsilon} + \mu)}$
		\end{center}
		\begin{center}
			$
			\displaystyle
			\quad
			\partial_t \hat{f_0} = \frac{1}{n_e} \partial_t f_0 - \mu \hat{f_0} = 0$ 
			$
			\displaystyle
			\quad
			\implies
			\frac{E^2}{3} \partial_{v} \of{\frac{\partial_v \hat f_0}{(n_0 \varepsilon^{1/2}\gamma \sigma\of{\varepsilon} + \mu)}} + \frac{2E^2}{3v} \frac{\partial_v \hat{f_0}}{(n_0 \varepsilon^{1/2}\gamma \sigma\of{\varepsilon} + \mu)} + \tilde C_0 \hat{f_0} = \mu \hat{f_0}$
		\end{center}
		\item Important to notice that we get a diffusion term in v-space. 
		\item Only solve for $\hat{f}_0$ and compute $\hat{f_1}$ from derivative of $\hat{f}_0$.
		\item The above is similar to the Bolsig+ derivation, with different models for electron growth rate $\mu$. 
	\end{itemize}
\end{frame}

\begin{frame}[fragile]
	\frametitle{Previous results vs. EEDF formulation}
	\centering

		\only<+>
		{
			\begin{itemize}
			\item Previous case \\
			\centering
			\begin{tabular}{cc}
			elastic & elastic + ionization \\
			\includegraphics[width=0.3\textwidth]{bspline_sp1_speed_g0.png} & 
			\includegraphics[width=0.3\textwidth]{bspline_sp1_speed_g0_g2.png} 
			\end{tabular}
			\end{itemize}
		}
		\only<+>
		{
			\begin{itemize}
				\item EEDF formulation \\
				\centering
				\begin{tabular}{cc}
					elastic & elastic + ionization \\
					\includegraphics[width=0.3\textwidth]{bspline_sp2_speed_g0_eedf.png} & 
					\includegraphics[width=0.3\textwidth]{bspline_sp2_speed_g0_g2_eedf.png} 
				\end{tabular}
			\end{itemize}
		}
		\begin{itemize}
			\item The diffusion term is needed to capture the tails accurately. 
		\end{itemize}
\end{frame}

\begin{frame}
	\frametitle{Observations}
	\begin{itemize}
		\item We need diffusion to stabilize tail oscillations.
		\item Global polynomials (Maxwell, and others)
		\begin{itemize}
			\item Pros
			\begin{itemize}
				\item Spectral convergence
				\item Work well with highly smoothed cross-sections
			\end{itemize}
			\item Cons
			\begin{itemize}
				\item With LXCAT data tails are not well resolved. 
				\item Struggle to capture sharp variations in $f$.
			\end{itemize}
		\end{itemize}
		\item Local approximations with B-Splines
		\begin{itemize}
			\item Pros
			\begin{itemize}
				\item Flexibility in quadrature, and knot placement. 
				\item Tails are better resolved compared to the global approximations. 
			\end{itemize}
			\item Cons
			\begin{itemize}
				\item Linear or quadratic convergence
			\end{itemize}
		\end{itemize}
	\end{itemize}
\end{frame}

\begin{frame}
	\frametitle{Validation with Bolsig+}
	\begin{itemize}
		\item We use Bolsig code, as a baseline for validation. 
		Hagelaar, G., BOLSIG+-Electron Boltzmann equation solver. 2013. URL: \url{https://www.bolsig.laplace.univ-tlse.fr}
		\item The code uses, fixed two term approximation $f(\vect{v},t) = f_0(v, t) + f_1(v,t)\cos v_\theta$
		\item We deploy (0,0) and (1,0) lm modes to match the expansion used in the Bolsig+.
		\item Compare steady-state solution, and QoIs computed with our approach. 
	\end{itemize}
\end{frame}

\begin{frame}
	\frametitle{Global basis: Maxwell polynomials}
	\centering
	\begin{tabular}{ccc}
	constant $\sigma$ & elastic & elastic + ionization \\
	\includegraphics[width=0.3\textwidth]{maxwell_speed_g0Const.png} & 
	\includegraphics[width=0.3\textwidth]{maxwell_speed_g0.png} & 
	\includegraphics[width=0.3\textwidth]{maxwell_speed_g0_g2.png}
	\end{tabular}
	\begin{itemize}
		\item Global polynomials are robust slow varying smooth data. 
		\item Not ideal with discontinuous cross section data. 
	\end{itemize}
\end{frame}

\begin{frame}
	\frametitle{Local basis: linear B-Splines}
	\centering
	\begin{tabular}{cc}
		elastic & elastic + ionization \\
		\includegraphics[width=0.3\textwidth]{bspline_sp1_speed_g0.png} & 
		\includegraphics[width=0.3\textwidth]{bspline_sp1_speed_g0_g2.png}
	\end{tabular}
	\begin{itemize}
		\item Linear splines based discretized operators, are equivalent to central difference scheme, hence no upwinding. 
		\item QoI (i.e., reaction rates) computed from B-splines were closer to Bolsig+ code, compare to the global polynomials. 
	\end{itemize}
\end{frame}