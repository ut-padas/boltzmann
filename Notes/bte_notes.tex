\documentclass{article}
\usepackage[utf8]{inputenc}
\usepackage{amsmath}
\usepackage{amssymb}
\usepackage{color}
\usepackage{amsmath}
\newtheorem{theorem}{Theorem}[section]
\newtheorem{definition}{Definition}[section]
\newcommand{\R}{\mathbb{R}}
\newcommand{\norm}[1]{\left\lVert#1\right\rVert}

\newcommand{\myint}{\int\limits}
\newcommand{\diff}[1]{\, d#1}
\newcommand{\vect}[1]{\mathbf{#1}}
\usepackage{mleftright}
\newcommand{\of}[1]{\mleft( #1 \mright)}
\newcommand{\ddt}[1]{\partial_t #1}
\newcommand{\vth}{v_\textrm{th}}
\newcommand{\reals}{\mathbb{R}}
\newcommand{\RR}{\mathbb{R}}
\usepackage{pgfplots}
\pgfplotsset{compat=newest}
\usepackage{float}
\usepackage{algorithm}
\usepackage{algpseudocode}

\title{Scalable deterministic numerical methods for the Boltzmann equation}
%\author{Milinda Fernando}
%\date{February 2014}

\begin{document}

%\begin{titlepage}
\maketitle
%\end{titlepage}

\section{Boltzmann Equation summary}
The Boltzmann equation is a nonlinear integro-differential equation that describes the evolution of the particle density function $f : \R^3 \times \R^3 \times \R \rightarrow \R^{+}_0$. Independent variables for the above density function denotes, time ($t$), position vector ($x$), and the velocity ($v$). For a fixed time $t$, $f(x,v,t)dx dv$ represent the number of particles for the phase space volume element $dxdv$. 

We can write the Boltzmann equation in the absence of external force field can be written as below. 
\begin{equation}
    \partial_t f + v\cdot \nabla_x f = \frac{1}{\epsilon}Q(f,f) \text{ for } x,v \in \R^3 \label{eq:be}
\end{equation}
In the presence of external force field (e.g., electromagnetic field) $L(x,t) : \R^3 \times \R \rightarrow \R$, the full Boltzmann equations can be written as, 
\begin{equation}
    \partial_t f + v\cdot \nabla_x f  + L \cdot \nabla_v f = \frac{1}{\epsilon}Q(f,f) \text{ for } x,v \in \R^3 \label{eq:be_full}
\end{equation}

The $\epsilon$ is known as the Knudsen number (dimensionless number, $\epsilon > 0$), $Q$ is the collision operator. 

\textbf{Note}: Unless stated otherwise, $v^\prime,v_*^\prime$ denotes the post-collision velocities and $v,v_*$ denotes the pre-collision velocities of a binary collision.

The collision operator (i.e., capture the physics of the collisions) can be broken up to two main parts, 
\begin{enumerate}
    \item Gain term ($C^+(f,f)$) : For a fixed $(t,x)$ how many particles created with velocity $v$.
    \item Loss term ($C^-(f,f)$) : For a fixed $(t,x)$ how many particles are lost with velocity $v$.
\end{enumerate} which are defined as follows. 

\subsection{Multi-species binary collisions}
\label{sec:multispecies_collissions}
Let $f_e,f_o$, denotes the distribution functions for electrons and heavy particles respectively. 



% \begin{align}
%     C^{+}(f,f) &= \int_{\R^3}\int_{S^2} B(|v-v_*|,\omega) f(v^\prime)f(v_*^\prime) d\omega dv_* \\
%     C^{-}(f,f) &= \int_{\R^3}\int_{S^2} B(|v-v_*|,\omega) f(v)f(v_*) d\omega dv_* 
% \end{align}

% The collision operator then defined as, 
% \begin{equation}
%     \begin{split}
%         C(f,f) = C^{+}(f,f) - C^{-}(f,f) %= \\ \int_{\R^3}\int_{S^2} B(|v-v_*|,\theta)(f(v^\prime)f(v_*^\prime) - f(v)f(v_*) ) d\omega dv_*    
%     \end{split}
% \end{equation}

\section{Electron - Ar collisions}
\label{sec:col_op_torch}

Assuming, that the neutral Ar atoms have $n_0\delta(0)$ distribution in the velocity space, where $n_0$ denotes the Ar density, and collision kernel $B$ approximation from experimental data, the above collusion operator simplifies to,
\begin{equation}
    C(f) = n_0 \int_{S^2} (f(v^\prime) - f(v)) \norm{v} \sigma(\norm{v},\omega) d\omega
\end{equation}

\subsection{Total and differential cross section}
For a given total cross section value, the differential cross section, can be computed as follows\cite{vahedi1995monte}, where $\varepsilon= \frac{1}{2}mv^2$
\begin{equation}
    \sigma(\varepsilon,\chi) = \frac{\sigma(\varepsilon)\varepsilon}{4\pi (1 + \varepsilon \sin^2(\chi/2))\ln(1+\varepsilon)} 
\end{equation}

\subsection{Post-collision velocity}
The scattering velocity direction is computed based on using following notations. Assumes vector coordinates w.r.t. basis $\hat{e_i},\hat{e_j}$ and $\hat{e_k}$.

\begin{itemize}
    \item $v_0$, $\hat{v_0}$ : pre-collision velocity, unit vector along $v_0$
    \item $v_1$, $\hat{v_1}$ : scattered (post-collision) velocity, unit vector along $\boldmath{v_1}$
    \item $\chi$ : scattering angle, i.e., the angle between vectors, $\hat{v_1}$ and  $\hat{v_0}$
    \item $\phi$ : angle between $v_1$ projection onto $v_0 \times(v_0 \times e_i)$, $v_0 \times e_i$ plane, and vector $v_0 \times(v_0 \times e_i)$
    \item $\theta$ : angle between $\hat{v_0}$ and $\hat{e_i}$, i.e., $cos\theta = \hat{v_0} \cdot \hat{e_i}$
\end{itemize}
The angle $\theta$ can be computed from $v_0$, and $\chi,\phi$ are taken to represent the solid angle for the collision event. The scattered velocity can be decomposed along the orthonormal basis vectors, $\hat{E_0}=\hat{v_{0}}$, $\hat{E_1}= \cfrac{\hat{v_0}\times e_i}{\sin\theta}$, and $\hat{E_2}= \hat{v_0}\times \hat{E_1}$. Therefore, the scattered direction unit vector can be written, 

\begin{equation}
    \hat{v_1} = \cos\chi \hat{v_0} + \sin\chi sin\phi (\frac{\hat{v_0}\times \hat{e_i}}{\sin\theta}) + \sin\chi cos\phi (\hat{v_0}\times \frac{\hat{v_0}\times \hat{e_i}}{\sin\theta}) \label{eq:scatter}
\end{equation}
We can see that, $\hat{v_1}\cdot \hat{v_1}=1$, and $\hat{v_1} \cdot \hat{v_0}= \cos\chi$.
When $\theta = 0 $, we can pick $\hat{E_0}=\hat{e_i}$, $\hat{E_1}=\hat{e_j}$ and $\hat{E_2}=\hat{e_k}$ as the basis to derive the scattering direction. 

In spherical coordinates, for a given incident vector $(v_r,\theta,\phi)$ that is not parallel to $\hat{e_i}$,  and scattering angle $(\chi,\gamma)$, we can compute the direction of the scattered particle as 

\begin{align}
    \theta^\prime &= \tiny \left\{\cos ^{-1}\left(\frac{\cos (\theta ) \left(\cos (\gamma ) \sin (\theta ) \sin (\chi ) \cos (\phi )+\cos (\chi ) \sqrt{1-\sin ^2(\theta ) \cos ^2(\phi )}\right)-\sin (\gamma ) \sin (\theta ) \sin (\chi ) \sin (\phi )}{\sqrt{1-\sin ^2(\theta ) \cos ^2(\phi )}}\right)\right\} \\
    \phi^\prime &= \tan ^{-1}\left(\frac{\sin (\chi ) \left(\cos (\gamma ) \sin ^2(\theta ) \sin (\phi ) \cos (\phi )+\sin (\gamma ) \cos (\theta )\right)+\sin (\theta ) \cos (\chi ) \sin (\phi ) \sqrt{1-\sin ^2(\theta ) \cos ^2(\phi )}}{\sin (\theta ) \cos (\chi ) \cos (\phi ) \sqrt{1-\sin ^2(\theta ) \cos ^2(\phi )}-\cos (\gamma ) \sin (\chi ) \left(\sin ^2(\theta ) \sin ^2(\phi )+\cos ^2(\theta )\right)}\right)
\end{align}

If the incident vector is parallel to the $\hat{e_i}$, we can compute the above, 
\begin{align}
    \theta^\prime &=\left\{\cos ^{-1}(\cos (\gamma ) \sin (\chi ))\right\} \\
    \phi^\prime &=\tan ^{-1}(\sin (\gamma ) \tan (\chi )) \\
\end{align}




\subsection{G0 : $e + Ar \rightarrow e + Ar$ collision operator}
\begin{equation}
    C_{G0}(f) = n_0 \int_{\chi} \int_{\phi} (f(v_1) - f(v_0)) \norm{v_0} \sigma_{G0}(\norm{v_0},\omega) \sin\chi d\phi d\chi
\end{equation}

Let $\varepsilon_0 = 1/2 m \norm{v_0}^2_2$ , $\varepsilon_1 = 1/2 m \norm{v_1}^2_2$, for inelastic collisions the energy lost, modeled based on \cite{vahedi1995monte}, (relative energy loss)
\begin{equation}
    \Delta \varepsilon = \frac{2m(1-cos\chi)}{M}
\end{equation} where, $m,M$ denotes the mass of the electron and the argon atom. Therefore the magnitude of the scattered velocity, can be written as, 
\begin{equation}
    \norm{v_1} = \norm{v_0} \sqrt{1- \frac{2m(1-cos\chi)}{M}} \label{eq:g0_eloss}
\end{equation} where the direction of $v_1$ is specified by $\hat{v_1}$ in (\ref{eq:scatter}).

\subsection{G1 : $e + Ar \rightarrow e + Ar^*$ collision operator}
\begin{equation}
    C_{G1}(f) = n_0 \int_{\chi} \int_{\phi} (f(v_1) - f(v_0)) \norm{v_0} \sigma_{G1}(\norm{v_0},\omega) \sin\chi d\phi d\chi
\end{equation}
Let $\varepsilon_{exc}$ be the energy threshold to trigger an excitation reaction, then we can write, 
\begin{align}
    \frac{1}{2} m v_0 ^2  - \varepsilon_{exc} &= \frac{1}{2} m v_1 ^2 \\
    \norm{v_1} &= \sqrt{\norm{v_0}^2 - \frac{2\varepsilon_{exc}}{m}}
\end{align} where the direction of $v_1$ is specified by $\hat{v_1}$ in (\ref{eq:scatter}). Note that, we use excitation threshold of $\varepsilon_{exc}=11.5eV$.

\subsection{G2 : $e + Ar \rightarrow e + Ar^+ + e$ collision operator}
\begin{itemize}
    \item $v_1$ : velocity of the scattered electron
    \item $v_2$ : velocity of the ejected electron from Ar. 
\end{itemize}
\begin{equation}
    C_{G2}(f) = n_0 \int_{\chi} \int_{\phi} (f(v_1) + f(v_2) - f(v_0)) \norm{v_0} \sigma_{G2}(\norm{v_0},\omega) \sin\chi d\phi d\chi
\end{equation}

Let $\varepsilon_{ion}$ be the energy threshold for the ionization reaction, then as in \cite{vahedi1995monte} we split the $\varepsilon_0-\varepsilon_{ion}$ equally among scattered and the ejected electron. i.e., $\varepsilon_1 = 0.5 (\varepsilon_0-\varepsilon_{ion})$, $\varepsilon_2 = 0.5 (\varepsilon_0-\varepsilon_{ion})$. Therefore, we can derive the velocity magnitudes of the scattered and ejected electrons as follows. 
\begin{align}
    \norm{v_1} &= \sqrt{\frac{1}{2}\norm{v_0}^2 - \frac{\varepsilon_{ion}}{m}}\\
    \norm{v_2} &= \sqrt{\frac{1}{2}\norm{v_0}^2 - \frac{\varepsilon_{ion}}{m}}
\end{align}
The direction of $v_1$ is given by $\hat{v_1}$ as in (\ref{eq:scatter}) and the direction of the $v_2$ derived based on the momentum conservation, assuming the momentum change in the Ar atom is negligible. 
\begin{align}
    m v_0 + M v &= Mv + m v_1 + m v_2  \\
    \hat{v_2} &= \frac{v_0 - v_1}{\norm{v_0 - v_1}}
\end{align}



\section{Maxwell-Boltzmann distribution (Maxwellian)}
\label{sec:maxwellian}
A particularly important velocity distribution function is the Maxwell-Boltzmann distribution, or Maxwellian. It describes the spread of velocities for a gas which is in thermal equilibrium. Maxwellian in $d$ dimensional velocity space can be written as,

\begin{equation}
    M(v) = A \exp(-\frac{mv^2}{2k_BT})
\end{equation}
using the number density equation, we can derive the coefficient $A$ as follows, 
\begin{equation}
M(v) = \frac{n}{(\sqrt{\pi}v_{th})^d} \exp{(-(\frac{v}{v_{th}})^2)}    
\end{equation} where, $v_{th}$ defined as, 
\begin{equation}
    v_{th} = \sqrt{\frac{2kT}{m}}
\end{equation}



%\section{Splitting the Boltzmann}
%\label{sec:split}

\section{Collision Operator}
\label{sec:collision_operator}
The term $v\cdot \nabla_x f$ makes the distribution $f(v)$ at fixed $(t,x)$ coupled with $f(v)$ defined in neighboring $x$. In operator splitting methods, Boltzmann equation is split in to ``transport'' and ``collision'' part. First we are going to focus on the collision part, given by, 
\begin{equation}
    \partial_t f = \frac{1}{\epsilon}C(f,f) \label{eq:col_op}
\end{equation}
% Note that without the transport term, for fixed $(t,v)$, for $x\neq x_*$ $f(x)$ and $f(x_*)$ can be evolved independently decoupled way. \emph{This allows, spatially subdivision based parallelization algorithms for the Boltzmann equation.}

% \subsection{Properties of the collision operator}
% \label{subsec:collision_op_properties}
% For a given $\R^d$ velocity space,
% \begin{theorem}
%     The Boltzmann collision operator conserves few of it moments, namely, mass, momentum, and energy.
%     \begin{align}
%         \int_{\R^d} Q(f,f) dv  = 0 \\
%         \int_{\R^d} Q(f,f) v dv  = 0 \\
%         \int_{\R^d} Q(f,f) |v|^2dv  = 0 \\
%     \end{align}
% \end{theorem}

% The solution for (\ref{eq:col_op}) evolves towards steady state, defined as the Maxwellian, which is given by, 
% \begin{equation}
%     f_{\infty}(v) = \frac{\rho}{(2\pi T)^{d/2}} \exp{-\frac{|V-v|^2}{2T}}
% \end{equation} where, the Maxwellian depends on the quantities computed from the initial distribution. 

% \begin{itemize}
%     \item \textbf{Density} : $\rho = \int_{\R^d} f(v) dv$ 
%     \item \textbf{Mean/Bulk velocity}: $V=\frac{1}{\rho} \int_{\R^d} v f(v) dv$ 
%     \item \textbf{Temperature}: $T= \frac{1}{3\rho} \int_{\R^d} |V-v|^2 f(v) dv$
% \end{itemize}

\section{0D space collision operator for electron-neutral collisions}
\label{sec:electron_neutral_pc}
Let $f_e(v,t)$ be the electron density and $f_o(v,t)=\delta(v)$ for all $(t,x)$. Then the evolution of the $f_e$ can be written as, 
\begin{equation}
    \partial_t f_e(v,t) = C(f_e,f_0)
\end{equation}
Let $(v^\prime,v_*^\prime) \rightarrow (v,v_*)$ be the pre and post collision velocities. The assumption $f_o(v,t)=\delta(v)$, implies neutral particles are mostly centered at velocity $0$. We can write the pre collision velocities as, $v^\prime=v^\prime(v,\omega)$. 

With these assumptions, the generic collision operator, for fixed time $t$ can be simplified for follows. 
\begin{align}
    C(f_e,f_0) &= \int_{\R^3}\int_{S^2} B(|v-v_*|,\omega)(f_e(v^\prime)f_0(v_*^\prime) - f_e(v)f_0(v_*) ) d\omega dv_*
\end{align}
Change of the integral order (assumes that the integral is finite) with properties of the Dirac's delta function, we can write, 
\begin{align}
    C(f_e,f_0) &= \int_{S^2} B(|v|,\omega)(f_e(v^\prime) - f_e(v)) d\omega
\end{align}

Then we can write the final evolution equation as, 
\begin{equation}
    \partial_t f_e(v,t) = \int_{S^2} B(|v|,\omega)(f_e(v^\prime,t) - f_e(v,t)) d\omega
\end{equation}

Since we know the above evolution will reach the Maxwellian at $t\rightarrow \infty$, we approximate $f_e(v,t)$ as follows, where $M(v)$ denotes the Maxwellian, 
\begin{equation}
    f_e(v,t) = M(v)[1 + h(v,t)] \label{eq:maxwelian}
\end{equation}
The idea is that when $t\rightarrow \infty$ the time dependent, $h(v,t)\rightarrow 0$.
Assuming that $\phi(v)$ is our test function with required properties. We can write the variational form for the above as, 
\begin{equation}
    \frac{\partial}{\partial t} \int_{\R^3} f_e(v,t) \phi(v) dv = \int_{\R^3} \int_{S^2} B(|v|,\omega)(f_e(v^\prime,t) - f_e(v,t)) \phi(v) d\omega dv \label{eq:0d_wf}
\end{equation}

Let $P_i(v)$ be orthonormal polynomial basis with the weighted inner product in the velocity space, where $w(v)$ denotes tha weight function. 
\begin{align}
    \int_{V} w(v)P_i(v)P_j(v) dv &= k_i\delta_{ij}
\end{align}

Assuming finite dimensional expansion for fixed time $t$, on $f(v,t)$, we can write, 
\begin{equation}
    f(v,t) \approxeq \bar{f}(v,t) = M(v)\sum_{j=0}^{N_v} f_j(t) P_j(v) \label{eq:basis_expansion}
\end{equation}

% Substituting (\ref{eq:basis_expansion}) to (\ref{eq:0d_wf}) we can write, 
% \begin{align}
%     \partial_t \int_{\R^3} w(v)h(v,t)\phi(v) dv &=  \int_{\R^3} \int_{S^2} (M(v^\prime) - M(v)) \phi(v) B(|v|,\omega) d\omega dv\\
%     &+\int_{\R^3} \int_{S^2} (M(v^\prime)h(v^\prime,t) - M(v)h(v,t)) B(|v|,\omega) \phi(v) d\omega dv
% \end{align}
For the above, By substituting, basis expansion for $f(v,t)$ we can write, 
\begin{align}
    \partial_t \int_{\R^3} M(v)\sum_{j=0}^{N_v} f_j(t) P_j(v) \phi(v) dv &= \\\int_{\R^3} \int_{S^2} M(v^\prime)\sum_{j=0}^{N_v} f_j(t) P_j(v^\prime) & B(|v|,\omega) \phi(v) d\omega dv \nonumber \\
    -\int_{\R^3} \int_{S^2} M(v)\sum_{j=0}^{N_v} f_j(t) P_j(v) &B(|v|,\omega) \phi(v) d\omega dv
\end{align}
By choosing $\phi(v) = P_i(v)$, we can further simplify, 
\begin{align}
    \text{diag}(k_i^\prime)\partial_t{f_i} = \sum_{j=0}^{N_v} L_{ij} f_j(t)
\end{align} where, 
\begin{equation}
    L_{ij} = \int_{\R^3} \int_{S^2} (M(v^\prime) P_i(v)P_j(v^\prime)  - M(v) P_i(v) P_j(v) )  B(|v|,\omega) d\omega dv
\end{equation}
\begin{equation}
    k_{i}^\prime = \int_{\R^3} M(v) P_i(v)^2 dv
\end{equation}

\subsection{Hermite Polynomials}
Let $H_k (x)$ be the sequence of Hermite polynomials, defined on $(\infty,infty)$ and are orthogonal with respect to the weight function $w(x) = \exp(-x^2/2)$. More precisely we can write, 
\begin{equation}
    \int_{-\infty}^{\infty} w(x) H_i(x) H_j(x) dx = \sqrt{2\pi} n! \delta_{ij}
\end{equation}where the polynomials can be generated using, 
\begin{equation}
    H_k(x) = (-1)^k \exp(x^2/2) \frac{d^k}{dx^k} \exp(-x^2/2)
\end{equation}. The corresponding Gauss-Hermite quadrature can be defined as, 
\begin{equation}
    \int_{-\infty}^{\infty} w(x) f(x) \approx \sum_{q=1}^{n} w_i f(x_i)
\end{equation} where, $x_i$ are the roots of $H_n(x)$, and $w_i$ are the corresponding weights for the Gaussian quadrature. 


% \subsection{Discretization of the integrals}
% \label{subsec:discretization_integrals}
% To perform discretization of the integrals in the velocity space we need to assume a finite bounded domain for the velocity space. 

% \textbf{Note}: The integrand will decay exponentially with increasing values of $v$, due to the $M(v)$. Hence, assume velocity integral is evaluated on the bounded subspace $V\subset \R^3$.

% \textbf{Integral over $S^2$}: 
% \begin{align}
%     I(g) &= \int_{S^2} g(\omega) d\omega\\
%     I(g) &= \int_{0}^{2\pi} \int_{0}^{\pi} g(\theta,\phi) \sin\theta d\theta d\phi 
% \end{align}

% Let $M_s$ denotes the number of $\theta_i$ chosen such that, $cos\theta_i,w_i$ are the Gauss-Legendre nodes and weights on $[-1,1]$. $\phi_j = \frac{j\pi}{m}$  are evenly spaced in $[0,2\pi]$. Then the above quadrature accurate up to polynomials order of $2M_s-1$.

% \begin{equation}
%     I(g) = \frac{\pi}{m} \sum_{q=1}^{2M_s} \sum_{p=1}^{M_s} w_p g(\theta_p,\phi_q) \label{eq:sphere_quadrature}
% \end{equation}

% % \textbf{Integral over velocity space ($\R^3$)}: Note that, when $|v|\rightarrow \infty$, $M(v)\rightarrow 0$, hence the integrand decays exponentially. Let's choose, bounded domain $V$, such that $V\subset \R^3$, such that, integral $\int_{\R^3\setminus V} M(v) \square dv < \epsilon$, where $\epsilon >0$. 

% \textbf{Integral over velocity space ($\R^3$)}: Let $x_r,w_r$ be the quadrature points, and weights in $\R^3$. For each velocity quadrature point and selected polynomial $P_j$, we can write the spherical quadrature as follows.  
% \begin{align}
%     Q(v_r)^{j} = \frac{\pi}{m} \sum_{q=1}^{2M_s} \sum_{p=1}^{M_s} &w_p (M(v^\prime(V_r,\theta_p,\phi_q))P_j(v^\prime(V_r,\theta_p,\phi_q))  -\\& M(v_r) P_j(v_r) ) \sin\theta_p B(|v_r|,\theta_p,\phi_k) \nonumber
% \end{align}
% Each $L_{ij}$ matrix entry can be computed as,  
% \begin{equation}
%     L_{ij} = \sum_{r} w_r Q(v_r)^{j}/w(v_r)
% \end{equation}




% Using the cartesian product on 1D basis functions, let's construct the basis functions in 3D. If we have $(N_v +1)$ coefficients in 1D, then we will have $(N_v+1)^3$ coefficients in 3D. Weights and quadrature points can be constructed the similar way for the domain $V$.  

% Then the above integral can be approximated as, 
% \begin{equation}
%     L_{ij} = \sum_{r=1}^{(N_v+1)^3} w_r Q(v_r)
% \end{equation} where, 
% \begin{align}
%     Q(v_r) = \frac{\pi}{m} \sum_{q=1}^{2M_s} \sum_{p=1}^{M_s} &w_p (M(v^\prime(V_r,\theta_p,\phi_q)) P_i(v_r)P_j(v^\prime(V_r,\theta_p,\phi_q))  -\\& M(v_r) P_i(v_r) P_j(v_r) ) \sin\theta_p B(|v_r|,\theta_p,\phi_k) \nonumber
% \end{align}

%Therefore the cost of computing the $L$ matrix would be $\mathbb{O}()$




% \begin{definition}{Hermite polynomials}
    
% \end{definition}

% \section{Time discretization}
% \label{sec:time_discretization}

% \begin{itemize}
%     \item How to ensure non-negativity of $f_e$ ? 
%     \item Implicit or Explicit ? 
% \end{itemize}

\section{Maxwellian polynomials with spherical coordinates}

For the electron-Boltzmann equation, the generalized weak form of the collision operator can be written as follows (note: velocity should be normalized by the thermal velocity).
\begin{equation}
    L_{ij} = n_0 \int_{\R^3} \int_{S^2} (M(v^\prime) P_i(v)P_j(v^\prime)  - M(v) P_i(v) P_j(v) )  B(|v|,\omega) d\omega dv \label{eq:cOp}
\end{equation}

\begin{equation}
    M_{ij} = \int_{\R^3} M(v) P_i(v)P_j(v) dv
\end{equation}

In spherical coordinates, the above becomes, 

\begin{align*}
    {L}_{k,l,m}^{p,q,s} &=n_0
    \myint_{0}^{+\infty} 
    v^2
    P^p \of{\frac{v}{\vth}} 
    \myint_{S^2}^{}
    \myint_{S^2}^{}
    B\of{v,\omega} Y^{qs}\of{v_\theta, v_\phi} 
     \times
    \\
    & \times
    \left(
    M\of{v^\prime} P_k\of{\frac{v^\prime}{\vth}} Y_{lm}\of{v_\theta^\prime, v_\phi^\prime}
    -
    M\of{v} P_k\of{\frac{v}{\vth}} Y_{lm}\of{v_\theta, v_\phi}
    \right)
    \diff{\omega}
    \diff{v_\omega}
    \diff{v} 
\end{align*}

\begin{align*}
    {M}_{k,l,m}^{p,q,s} &=
    \myint_{0}^{+\infty} 
    v^2
    M\of{v} P^p \of{\frac{v}{\vth}}  P_k \of{\frac{v}{\vth}} \delta^{qs}_{lm}
    \diff{v} 
\end{align*}

\subsection{Tensorized computation of ${L}_{k,l,m}^{p,q,s}$}

\begin{align*}
    {L}_{k,l,m}^{p,q,s} &= {L^{+}}_{k,l,m}^{p,q,s} - {L^{-}}_{k,l,m}^{p,q,s}
\end{align*} where, 
\begin{align*}
    {L^{+}}_{k,l,m}^{p,q,s} = n_0 \int_{v_r} 
                               \int_{S^2(v_\theta,v_\phi)}
                               \int_{S^2(\chi,\gamma)} & 
                               v^2 M(v^\prime) P^p \of{\frac{v}{\vth}} Y^{qs}\of{v_\theta, v_\phi} \times \\ & P_k\of{\frac{v^\prime}{\vth}} Y_{lm}\of{v_\theta^\prime, v_\phi^\prime} |v|\sigma(|v|,\chi) d\omega d\omega_v dv \\
    {L^{-}}_{k,l,m}^{p,q,s} = n_0 \int_{v_r} 
                              \int_{S^2(v_\theta,v_\phi)}
                              \int_{S^2(\chi,\gamma)} & 
                              v^2 M(v) P^p \of{\frac{v}{\vth}} Y^{qs}\of{v_\theta, v_\phi} \times \\ & P_k\of{\frac{v}{\vth}} Y_{lm}\of{v_\theta, v_\phi} |v|\sigma(|v|,\chi) d\omega d\omega_v dv
\end{align*}

The list of tensors that can be precomputed
\begin{itemize}
    \item $V_r$ - quadrature points on the radial direction (incident velocities)
    \item $W_r$ - quadrature weights on the radial direction
    \item $V_\theta$ - quadrature points on the polar direction 
    \item $W_\theta$ - quadrature weights for theta
    \item $V_\phi$ - quadrature points on the azimuthal direction 
    \item $S_\chi$ - quadrature points on the scattering angle
    \item $S_\gamma$ - quadrature points on the azimuthal angle (for scattering direction)
    \item $W_\chi$ - quadrature weights
    \item $\sigma_{r\chi}$ - differential cross section tensor (rank 2)
    \item $Y_{lm}^{\theta\phi}$ - $lm$-mode spherical harmonic function evaluated at $(V_\theta, V_\phi)$. The sparse version (i.e., for given $l$ mode not selecting all the $m$ modes) , but generally can be considered as rank 4 tensor. 
    \item $M_r$ - Maxwellian times $v_r$ evaluated at $V_r$
    \item $P_{kr}$     - $k^{th}$ Maxwell polynomial evaluated at the $V_r$ $r^{th}$ location.
\end{itemize}
Notation : Same index up-down denotes contraction, $\otimes$ for kronecker product, same level index, same index (i.e., up-up, down down) denotes the element-wise multiplication. 
The total cross section $\sigma_r$, can be written as, 
\begin{align}
    \sigma_r &= \frac{\pi}{|S\chi|} \sigma_{r,\chi} W^\chi
\end{align}

The weighted spherical harmonic tensor, 
\begin{align}
    \tilde{Y}^{qsr}_{\theta\phi} = \mleft({Y}^{qs}_{\theta\phi} W_\theta \frac{\pi}{|V\chi|} \mright) 
\end{align}

Then we can write, 
\begin{align}
    {L^{-}}_{k,l,m}^{p,q,s} &= ((P^p_r W_r \sigma_r) (P^r_k M^r) ) \otimes Y_{lm}^{\theta\phi} \tilde{Y}^{qs}_{\theta\phi} \\
    {L^{-}}_{k,l,m}^{p,q,s} &= ((P^p_r W_r \sigma_r) (P^r_k M^r) ) \otimes \delta^{qs}\delta_{lm} %Y_{lm}^{\theta\phi} \tilde{Y}^{qs}_{\theta\phi} 
\end{align}


More additional tensors that we need to compute the $L^{+}$ component. 

\begin{itemize}
    \item $S^{r\theta\phi\chi\gamma}_{r^\prime\theta^\prime \phi^\prime}$ : Scattering velocity tensor, for each $v=(r,\theta,\phi)$ and scattering solid angle $(\chi,\gamma)$ computes $(r^\prime,\theta^\prime,\phi^\prime)$ scattered or newly created particle velocity (i.e., in G2 ejected electron). This is a rank 8 tensor where it might be too expensive to compute. For cases $G0$, $G1$ we can compute $S^{r\theta\phi\chi\gamma}_{r^\prime\theta^\prime \phi^\prime} = {S^r}_{r^\prime} \otimes S^{\theta\phi\chi\gamma}_{\theta^\prime \phi^\prime}$ since radial component only depends on energy, while for reactions like $G2$ it's depends on both energy and direction (i.e., for momentum conservation).
    \item $P^{r\theta\phi\chi\gamma}_{k}$ - radial polynomial evaluated at differed velocity for given incident particle ($r,\theta,\phi,\chi,\gamma$)
    \item $M^{r\theta\phi\chi\gamma}$ - Maxwellian times $v_r$ evaluated for the differed particle for a given incident particle ($r,\theta,\phi,\chi,\gamma$)
    \item $Y^{r\theta\phi\chi\gamma}_{lm}$ - $lm$ spherical harmonic mode evaluated differed particle direction for a given incident particle ($r,\theta,\phi,\chi,\gamma$)
    \item $\sigma^{r\theta\phi\chi\gamma}$ - differential cross section broadcasted on scattering cross section angles. 
    \item $B^{r\theta\phi}_{pqs}$ - $pqs$ basis evaluated at the incident grid.              
\end{itemize}

For the general case, of the differed particle, (i.e., differed particle all velocity components are functions of $r,\theta,\phi,\chi,\gamma$) we can write the following. Note, $A$ is obtained for contraction on the $\gamma$ azimuthal of angle of the scattered particle, $B$ is obtained with contraction on the polar angle of the scattering direction, $C$ is obtained contraction on $(\theta,\phi)$ for the velocity space angular directions, and finally $L^{+}$ obtained using radical direction contraction. 

\begin{align}
    A^{r\theta\phi}_{klm}   &=  ( P^{r\theta\phi\chi\gamma}_{k} M^{r\theta\phi\chi\gamma} Y^{r\theta\phi\chi\gamma}_{lm}) W{\gamma} W_{\chi} \\
    {L^{+}}_{k,l,m}^{p,q,s} &=  B^{r\theta\phi}_{pqs} A^{r\theta\phi}_{klm} W_\phi W_\theta W_r
    % B^{r\theta\phi}_{klm}      & = \frac{\pi}{|S_\gamma|} \sigma^{r\chi} (W_\chi A^{r\theta\phi}_{\chi klm}) \\
    % C^{rqs}_{klm}              & = \tilde{Y}^{qs}_{\theta\phi} B^{r\theta\phi}_{klm} \\
    %{L^{+}}_{k,l,m}^{p,q,s}     & = {P^p}_r C^{rqs}_{klm}
\end{align}

\subsection{Distribution moments in basis expansion}
Let $f(v) = M(v_\alpha) \sum_{klm} f_{klm} \phi_{klm}(v_\alpha)$, where $v_\alpha = v/\alpha$. Then we can write the zeroth moment (number of species particles) of the distribution, 
\begin{align}
    n_e & = \int_{R^3} f(v) dv \\
\end{align} which can be written as, 
\begin{align}
    n_e & = q^{klm} f_{klm}  
\end{align} where, 
\begin{align}
    q_{klm} & = \int_{\R^3} M(v_\alpha) \phi_{klm}(v_\alpha) dv = 0  \text{ if } klm \neq 000 \\
    q_{000} & = \int_{\R^3} M(v_\alpha) \phi_{klm}(v_\alpha) dv = \frac{n}{4\pi} 
\end{align}



\subsection{Varying thermal velocity}
Let $\alpha,\beta$ be two thermal velocities, and their corresponding normalized velocity be $v_\alpha=v/\alpha,v_\beta=v/\beta$.
Let $f(v) = M(v_\alpha) h(v_\alpha)$. Let $h^{(\alpha)}$ be the basis coefficients w.r.t. the chosen basis. Let us try to expand $f(v)$ using maxwellian at different temperature say at $\beta$, let $h^{(\beta)}$ be the coefficients for the expansion.
\begin{equation}
    M^{(\beta)} h^{(\beta)} = W^{(\alpha)} h^{(\alpha)}
\end{equation} where, 
\begin{align}
    M^{(\beta)}_{ij} &= \int_{v} \int_{S^2} M(v_\beta) \phi_i(v_\beta) \phi_j(v_\beta) d\omega dv  = \frac{n}{4\pi} \delta_{ij} \\
    W^{(\alpha)}_{ij} &= \int_{v} \int_{S^2} M(v_\alpha) \phi_i(v_\beta) \phi_j(v_\alpha) d\omega dv   \label{eq:bchnage}
\end{align}

For $W_ij^{(\alpha)}$, let i=0, then we can write (Note : $\phi_0(v_\beta)=1$)
\begin{align*}
    W^{(\alpha)}_{0j} &= \int_{v} \int_{S^2} M(v_\alpha) \phi_0(v_\beta) \phi_j(v_\alpha) d\omega dv\\
    W^{(\alpha)}_{0j} &= \int_{v} \int_{S^2} M(v_\alpha) \phi_j(v_\alpha) d\omega dv = \frac{n}{4\pi} \delta_{0j}
\end{align*}
Therefore, we can write $h^{(\beta)}_0=h^{(\alpha)}_0$. There zeroth coefficient always match hence we should preserve zeroth moment of the distribution under basis change.  

\subsection{Bounds on the tails for the basis change}
The thermal velocity basis change operator is given by, \eqref{eq:bchnage}, using orthogonality in the angular directions we can only operate in the radial direction. Therefore, the above simplifies to following (note that we can consider scaling of the 1d Maxwellian)
\begin{align}
    W_{ij} &= \int_{\reals} M(v_\alpha) P_i(v_\beta) P_j(v_\alpha)dv \label{eq:basis_change}
\end{align}
Let $\epsilon >0$, $\alpha = \beta + \epsilon$. Then we can write the simplified $W_{ij}$ as, 
\begin{align}
    W_{ij} &= \frac{n}{\sqrt{\pi}^{3}}  \int_{\reals}  \exp(-v_\alpha^2) P_i (v_\alpha (1 + \frac{\epsilon}{\beta})) P_j(v_\alpha) \frac{dv}{\alpha} \\
\end{align}


\subsection{Mass conservation in the current discretization}
For any smooth distribution, $f(v) = M(v_\alpha) \sum_{klm} f_{klm} \phi_{klm}(v_\alpha)$, and we can write, 
\begin{align}
    \int_{\RR^3} f dv &= const.  \\
    \frac{d}{dt} \int_{\RR^3} f dv &= 0  \\
    \int_{\RR^3} \partial_t f  dv &= \int_{\RR^3} c(f) dv =0 \\
    % \partial_t f &= c(f) = n_0 \int_{S^2} (f(v^\prime) - f(v)) \norm{v} \sigma(\norm{v},\omega) d\omega \\
    % \frac{d}{dt} \int_{\RR^3} f dv =0  &\implies \int_{\RR^3} c(f) dv = 0 
\end{align}
We can further write, 
\begin{align}
    \int_{\RR^3} c(f) dv  &= \sum_{klm} f_{klm} \int_{\RR^3}\int_{S^2} \bigl( M(v^\prime/\alpha) \phi_{klm}(v^\prime/\alpha) -   M(v/\alpha) \phi_{klm}(v/\alpha) \bigr) \norm{v} \sigma(\norm{v},\omega) d\omega dv, \  \forall f(v,t) \\
    \implies & \int_{\RR^3}\int_{S^2} \bigl( M(v^\prime/\alpha) \phi_{klm}(v^\prime/\alpha) -   M(v/\alpha) \phi_{klm}(v/\alpha) \bigr) \norm{v} \sigma(\norm{v},\omega) d\omega dv =0  \forall klm
\end{align}
But the above is not satisfied, take const polynomial $\phi_{000}(x)=1/2\sqrt{\pi}$ with $v^\prime \neq v$.  

\newpage
\section{Numerical evaluation}
This section presents, a numerical evaluation of the spatially homogeneous Boltzmann equation without external force field. Recall, under the above assumptions we can write the collision source term as \eqref{eq:nr_be}, 
\begin{equation}
	\partial_t f_e(v,t) = \sum_{k} n_k C_k(f_e) \label{eq:nr_be}
\end{equation} where the $C_k$ denotes the collision operator for the $k^{th}$ collision. We consider the following experiments in the study. 
\begin{itemize}
\item G0: $e + Ar \rightarrow e + Ar $ 
\item G1: $e + Ar \rightarrow e + Ar^*$
\item G2: $e + Ar \rightarrow e + Ar^+ + e$ 
\end{itemize} The post collision velocities for the above collisions are summerized in \S\ref{sec:col_op_torch}.

\subsection{Eigenmodes of the collision operator}
For the assembled collision operator $L$, we construct the matrix, 
\begin{equation}
	E_t = Re\left( Q^{-1} \times  \text{diag}(\exp(\lambda_i t)) \times Q \right) 
\end{equation}, where $L=Q^{-1} \Lambda Q$ is the eigen decomposition of $L$. If $L$ is linear the solution at time $t$ can be written as a linear combination of the columns vectors of $E_t$ where the coefficients are determined by the initial condition of the ODE. (see figures \ref{fig:eig_g_sep} \& \ref{fig:eig_g02_g012})

\begin{itemize}
	\item For G0, first eigenmode ensures (the top right most figure in \ref{fig:eig_g_sep}) that if we start with Maxwellian the solution will stay at maxwellian for shorter periods in time. For longer periods the other modes will grow over time. The above is mainly due to the assembled collission operator is bound to the thermal velocity through the sampling of the experimental cross section data. 
	\item For G1 and G2, at 1eV $E_t$ is close to the identity matrix, since the probability of triggering a excitation or ionization reaction is low, at 1eV. 
	\item $E_t$ modes will depend on the background species densities, that scales the differential cross sections. 
\end{itemize}
\begin{figure}[!htbp]
	\includegraphics[width=0.32\textwidth]{fig/g0_Nr_7_eig_ev_1.00E+00_NI_1.00E+18_N0_3.22E+22_NE_1.00E+18.png}
	\includegraphics[width=0.32\textwidth]{fig/g1_Nr_7_eig_ev_1.00E+00_NI_1.00E+18_N0_3.22E+22_NE_1.00E+18.png}
	\includegraphics[width=0.32\textwidth]{fig/g2_Nr_7_eig_ev_1.00E+00_NI_1.00E+18_N0_3.22E+22_NE_1.00E+18.png}
	\caption{Eigen modes for the collision operator $G0,G1,G2$ (left most, middle, and the right most) evaluated at different time points. Collision operators are assembled under 1eV, $n_i=1e18,n_e=1e18, n_o = 3.22e22$.\label{fig:eig_g_sep}}
\end{figure}
\begin{figure}[!htbp]
	\includegraphics[width=0.48\textwidth]{fig/g02_Nr_7_eig_ev_1.00E+00_NI_1.00E+18_N0_3.22E+22_NE_1.00E+18.png}
	\includegraphics[width=0.48\textwidth]{fig/g012_Nr_7_eig_ev_1.00E+00_NI_1.00E+18_N0_3.22E+22_NE_1.00E+18.png}
	\caption{Eigen modes for the collision operator $G0+G2$ (left) and $G0+G1+G2$ (the right) evaluated at different time points. Collision operators are assembled under 1eV, $n_i = 1e18, n_e=1e18, n_o = 3.22e22$.\label{fig:eig_g02_g012}}
\end{figure}

\subsection{Handling changing temperature}
Due to the kinetic energy loss in $e-Ar$ collisions, the thermal velocity decreases with the time. Since we use quadrature points on the normalized velocity (i.e., $\frac{v}{v_th}$) we need transform the solution from one temperature to another. If the temperature drop is significant and if we don't have enough basis functions in the radial direction, the projection between temperature cause an aliasing effect and gives larger tails. 

Let $T_\alpha$ be the current temperature and we want to transform the solution to $T_\beta=T_\alpha (1+ \epsilon)$ where $\epsilon \in [-a,a]$ for $a>0$.

\begin{align}
W_{ij} &= \int_{\reals} M(v_\alpha) P_i(v_\beta) P_j(v_\alpha)dv \label{eq:basis_change}
\end{align}
Let $\epsilon >0$, $\alpha = \beta + \epsilon$. Then we can write the simplified $W_{ij}$ as, 
\begin{align}
W_{ij} &= \frac{n}{\sqrt{\pi}^{3}}  \int_{\reals}  \exp(-v_\alpha^2) P_i (v_\alpha (1 + \frac{\epsilon}{\beta})) P_j(v_\alpha) \frac{dv}{\alpha} \\
\end{align}

\begin{figure}[!htbp]
	\centering
	\includegraphics[width=0.48\textwidth]{fig/basis_1ev_grid.png}
	\includegraphics[width=0.48\textwidth]{fig/basis_1ev_tail.png}
	\caption{The left most figure shows the $f(v)$ expansion evaluated at uniform grid in the radial direction followed by normed difference computation for increasing $N_r$ polynomials. For a given Maxwellian distribution (i.e., $h^{\alpha}=1$) we compute the projection of the solution to $T_\beta=T_\alpha (1+\epsilon)$ with computed $W_{\beta\alpha}$, where $h^{\beta} = W_{\beta \alpha} h^{\alpha}$. The right most figure shows the tail of the computed $h^{\beta}$ with increasing polynomials in the radial direction.  \label{fig:basis_projection_error}
	}
\end{figure}
\begin{figure}[!hbtp]
	\centering
	\includegraphics[width=0.48\textwidth]{fig/basis_1ev_f_NR_3.png}
	\includegraphics[width=0.48\textwidth]{fig/basis_1ev_log_f_NR_3.png}
	
	\includegraphics[width=0.48\textwidth]{fig/basis_1ev_f_NR_7.png}
	\includegraphics[width=0.48\textwidth]{fig/basis_1ev_log_f_NR_7.png}
	
	\includegraphics[width=0.48\textwidth]{fig/basis_1ev_f_NR_15.png}
	\includegraphics[width=0.48\textwidth]{fig/basis_1ev_log_f_NR_15.png}

	\includegraphics[width=0.48\textwidth]{fig/basis_1ev_f_NR_31.png}
	\includegraphics[width=0.48\textwidth]{fig/basis_1ev_log_f_NR_31.png}
	\caption{ For each row the the right most figure shows the log scale to emphasis the variation in the tails of the distribution function. Each row corresponds to a temperature change indicated byt the $\epsilon$ parameter, and projected coefficient evaluated at a uniformly space grid of $10^5$ points in the range of $(0,6V_th)$. 
		\label{fig:basis_grid_plot}
	}
\end{figure}





\subsection{Assembling on the collision operator}
Let $N_r$ , $N_\theta$, and $N_\phi$ denotes the number of polynomials used in radial, polar and azimuthal direction. 

Let $Q_{r}, Q_{\theta}, Q_{\phi}$ be the quadrature points along radial, polar and azimuthal directions for the velocity space. $Q_{r} \times Q_{\theta} \times Q_{\phi}$ forms the quadrature grid for the electron collision incident events. 

Let $Q_{\chi},Q_{\gamma}$ be the quadrature points in polar and azimuthal direction in scattering solid angle. 

\begin{table}[H]
	\centering
	\begin{tabular}{|c|c|c|}
		\hline
		Grid & $(Q_\gamma,Q_\theta,Q_\phi)=(2,8,4)$ & $(Q_\gamma,Q_\theta,Q_\phi)=(4,16,8)$ \\
		\hline
		$Q_r=20,Q_\chi=8,$ & - & - \\
		$Q_r=40,Q_\chi=16,$ & 7.46E-01 & - \\
		$Q_r=51,Q_\chi=32,$ & 4.99E-03 & - \\
		$Q_r=51,Q_\chi=64,$ & 7.48E-09 & - \\
		$Q_r=51,Q_\chi=128,$ & - & 2.09E-12 \\
		\hline
	\end{tabular}
	\caption{Relative error in the normed difference in the assembled collision operator for $G0$ at 1ev for $N_r=31,N_\theta=1$, and $N_\phi=1$.}
\end{table}

\begin{table}[H]
	\centering
	\begin{tabular}{|c|c|c|}
		\hline
		Grid & $(Q_\gamma,Q_\theta,Q_\phi)=(2,8,4)$ & $(Q_\gamma,Q_\theta,Q_\phi)=(4,16,8)$ \\
		\hline
		$Q_r=20,Q_\chi=8,$ & - & - \\
		$Q_r=40,Q_\chi=16,$ & 1.16E+00 & - \\
		$Q_r=51,Q_\chi=32,$ & 3.46E-03 & - \\
		$Q_r=51,Q_\chi=64,$ & 1.77E-07 & - \\
		$Q_r=51,Q_\chi=128,$ & - & 1.65E-13 \\
		\hline
	\end{tabular}
	\caption{Relative error in the normed difference in the assembled collision operator for $G2$ at 1ev for $N_r=31,N_\theta=1$, and $N_\phi=1$.}
\end{table}

Therefore, for the convergence study it is reasonable to use a quadrature grid size of,$(Q_{r}, Q_{\theta}, Q_{\phi},Q_{\chi},Q_{\gamma}) =(51,8,4,64,2)$.

\subsection{G0 at (1eV)}
\begin{table}[H]
	\centering
	\begin{tabular}{|c|c|c|c|c|}
		\hline
		$N_r$ & $dt=1E-10$ & $dt=5E-11$ & $dt=2.5E-11$ & $dt=1.25E-11$\\
		\hline
		3   & \textcolor{red}{1.61605e-05} & 1.61614e-05  & 1.61619e-05  &  1.61621e-05 \\
		7   & 6.87078e-07 & \textcolor{red}{6.86688e-07}  &  6.86494e-07 &   6.864e-07 \\
		15  & 1.83438e-07 & 1.83005e-07  &  \textcolor{red}{1.8279e-07}  &  1.82685e-07 \\
		31  & 7.85306e-10 & 3.33874e-10  &  1.09496e-10 &   0 \\
		\hline
	\end{tabular}
	\caption{Relative error in temperature compared at matching time points taking $N_r=31,dt=1.25E-11$ as the baseline over time horizon $T$=1e-06 s for G0 collision operator. }
\end{table}

\begin{figure}[H]
	\centering
	\includegraphics[width=0.49\textwidth]{fig/g0_temp_convergence.png}
	\includegraphics[width=0.49\textwidth]{fig/g0_dt_1.25000000E-11_Nr_31.dat.png}
	\caption{Relative error in temperature (the left most figure) compared at matching time points taking $N_r=31,dt=1.25E-11$ as the baseline over time horizon $T$=1e-06 s for G0 collision operator. The right most figure shows the highest resolution, EEDF, mass growth, temperature evolution and the tail norm of the used basis functions.  }
\end{figure}


\subsection{G0 + G2 at (1eV)}
\begin{table}[H]
	\centering
	\begin{tabular}{|c|c|c|c|c|}
		\hline
		$N_r$ & $dt=1E-10$ & $dt=5E-11$ & $dt=2.5E-11$ & $dt=1.25E-11$\\
		\hline
		
		 3   & \textcolor{red}{1.40916e-05} & 1.40925e-05 &  1.40924e-05  &  1.40923e-05 \\
		 7   & 6.06432e-07 & \textcolor{red}{6.06074e-07} &  6.05866e-07  &  6.05765e-07 \\
		 15  & 1.54661e-07 & 1.5426e-07  &  \textcolor{red}{1.54052e-07}  &  1.5395e-07  \\
		 31  & 7.26707e-10 & 3.08953e-10 &  1.01324e-10  &  0  \\
		\hline
	\end{tabular}
\caption{Relative error in temperature compared at matching time points taking $N_r=31,dt=1.25E-11$ as the baseline over time horizon $T$=9e-07 s for G0 +G2 collision operator with enforced quasi neutrality.}
\end{table}

\begin{figure}[H]
	\centering
	\includegraphics[width=0.48\textwidth]{fig/g02_temp_convergence.png}
	\includegraphics[width=0.48\textwidth]{fig/g02_dt_1.25000000E-11_Nr_31.dat.png}
	\caption{Relative error (the left most figure) in temperature compared at matching time points taking $N_r=31,dt=1.25E-11$ as the baseline over time horizon $T$=9e-07 s for G0 + G2 collision operator. The right most figure shows the highest resolution EEDF, mass growth, temperature evolution and the tail norm of the used basis functions.  }
\end{figure}

\begin{figure}[H]
	\centering
	\includegraphics[width=0.8\textwidth]{fig/g02_mass_convergence.png}
	\caption{Relative error in number of electrons compared at matching time points taking $N_r=31,dt=1.25E-11$ as the baseline over time horizon $T$=9e-07 s for G0 + G2 collision operator. }
\end{figure}


%\subsection{G0: $e + Ar \rightarrow e + Ar $ (1eV)}
%This section presents a summary of the numerical experiments conducted on the different reactions. For the following convergence study, we study the $G0$ collision operator with energy loss modeled by \eqref{eq:g0_eloss}. 
%\begin{itemize}
%    \item Uniform time step size used, with collision operator reassembled at every time step. 
%    \item RK4 time integrator is used for the convergence study. 
%    \item At each step, the coefficient are projected to the correct temperature using the basis transformation in \eqref{eq:basis_change}
%    \item For all convergence runs, we use
%    \begin{itemize}
%    	\item $N_\theta = 2$ 2 polynomials in the polar angle direction.
%		\item $N_\phi   = 1$ 1 polynomials in the polar angle direction.
%		\item $Q_r=21, Q_\theta=16, \Q_phi=16, Q_\chi=64, Q\_gamma=16$, we
%    \end{itemize}
%\end{itemize}






\bibliographystyle{plain}
\bibliography{bte_notes.bib}
\newpage
\appendix

\section{Derivation of Collision operators}

\subsection{Binary reactions}

We start with reactions of the type
\begin{align*}
\text{Ar} + e \longrightarrow \tilde{\text{Ar}} + e
\end{align*}
where $\tilde{\text{Ar}} = \text{Ar}$ in the case of elastic collisions and $\tilde{\text{Ar}} = \text{Ar}^\ast$ in the case of excitation events. Let us denote the expressions that map given pre-collisional velocities $\vect{v}_e$, $\vect{v}_0$ to post-collisional velocities as 
\begin{align*}
\vect{v}_e^\text{post} &= \vect{v}_e^\text{post} \of{\vect{v}_e, \vect{v}_0, \vect{\omega}}
\\
\vect{v}_0^\text{post} &= \vect{v}_0^\text{post} \of{\vect{v}_e, \vect{v}_0, \vect{\omega}}
\end{align*} 
where $\vect{\omega} \in S^2$ is the vector defining along which directions velocities change in the reaction. Specifically, from the conservation of momentum and energy it can be derived that
\begin{align*}
\vect{v}_e^\text{post} &= \vect{v}_e + \frac{\alpha}{m_e}\vect{\omega},
\\
\vect{v}_0^\text{post} &= \vect{v}_0 - \frac{\alpha}{m_0}\vect{\omega},
\end{align*}
where
\begin{align*}
\alpha &= \frac{u + \sqrt{u^2 - 4 \Delta E \mu}}{2\mu},
\quad 
u = \vect{\omega} \cdot \left( \vect{v}_0 - \vect{v}_e \right),
\quad 
\mu = \frac{m_e+m_0}{2 m_e m_0}
\end{align*}
and $\Delta E$ denotes the energy loss during the reaction ($\Delta E = 0$ for elastic collisions). Note that if $\Delta E=0$, then
\begin{align*}
\vect{v}_e^\text{post} \of{\vect{v}_e, \vect{v}_0, \vect{\omega}} = \vect{v}_e,
\\
\vect{v}_0^\text{post} \of{\vect{v}_e, \vect{v}_0, \vect{\omega}} = \vect{v}_0,
\end{align*} 
for $\vect{\omega} \cdot \left( \vect{v}_0 - \vect{v}_e \right) < 0$, that is, no collision happens.

The inverse map (assuming it exists and well-defined) is denoted as 
\begin{align*}
\vect{v}_e^\text{pre} &= \vect{v}_e^\text{pre} \of{\vect{v}_e, \vect{v}_0, \vect{\omega}}
\\
\vect{v}_0^\text{pre} &= \vect{v}_0^\text{pre} \of{\vect{v}_e, \vect{v}_0, \vect{\omega}}
\end{align*} 
where $\vect{v}_e$, $\vect{v}_0$ now represent the post-collision velocities. Thus we have 
\begin{align*}
\vect{v}_e &= \vect{v}_e^\text{pre} \of{\vect{v}_e^\text{post} \of{\vect{v}_e, \vect{v}_0, \vect{\omega}}, \vect{v}_0^\text{post} \of{\vect{v}_e, \vect{v}_0, \vect{\omega}}, \vect{\omega}}
\\
\vect{v}_0 &= \vect{v}_0^\text{pre} \of{\vect{v}_e^\text{post} \of{\vect{v}_e, \vect{v}_0, \vect{\omega}}, \vect{v}_0^\text{post} \of{\vect{v}_e, \vect{v}_0, \vect{\omega}}, \vect{\omega}}
\end{align*} 

Let us denote the collision kernel of reaction as $B\of{\vect{v}_e, \vect{v}_0, \vect{\omega}}$ which is a function of pre-collision velocities $\vect{v}_e$, $\vect{v}_0$ and the direction of velocity change $\vect{\omega}$. The number of electrons with velocity $\vect{v}_e$ that will participate in the reaction and, thus, lost is given by 
\begin{align*}
C^- = \myint_{R^3} \myint_{S^2} B\of{\vect{v}_e, \vect{v}_0, \vect{\omega}} f_e\of{\vect{v}_e} f_0\of{\vect{v}_0} \diff{\vect{v}_0} \diff{\vect{\omega}}
\end{align*}
The number of electrons with the same velocity created in the reaction is given by
\begin{align*}
C^+ = \myint_{R^3} \myint_{R^3} \myint_{S^2} 
B\of{\vect{v}_e^\prime, \vect{v}_0^\prime, \vect{\omega}} 
f_e\of{\vect{v}_e^\prime} f_0\of{\vect{v}_0^\prime} 
\delta\of{\vect{v}_e^\text{post}\of{\vect{v}_e^\prime, \vect{v}_0^\prime, \vect{\omega}} - \vect{v}_e} 
\diff{\vect{v}_0^\prime} \diff{\vect{v}_e^\prime} \diff{\vect{\omega}}
\end{align*}
where we integrate over all possible pre-collision velocities $\vect{v}_e^\prime$, $\vect{v}_0^\prime$ but pick out only those that result in post-collision electron velocity $\vect{v}_e$ (thanks to the delta function). Note that in the expression for $C^-$ symbols $\vect{v}_e$, $\vect{v}_0$ have the meaning of pre-collision velocity, while in the expression for $C^+$ those are denoted by $\vect{v}_e^\prime$, $\vect{v}_0^\prime$. 

\textbf{Remark.} One could define $C^-$ in an analogous to $C^+$ way. That is, consider reactions for all possible pre-collision velocities $\vect{v}_e^\prime$, $\vect{v}_0^\prime$ but select only those that lead to loss of electrons with velocity $\vect{v}_e$
\begin{align*}
C^- = \myint_{R^3} \myint_{R^3} \myint_{S^2} 
B\of{\vect{v}_e^\prime, \vect{v}_0^\prime, \vect{\omega}} 
f_e\of{\vect{v}_e^\prime} f_0\of{\vect{v}_0^\prime} 
\delta\of{\vect{v}_e^\prime - \vect{v}_e} 
\diff{\vect{v}_0^\prime} \diff{\vect{v}_e^\prime} \diff{\vect{\omega}}.
\end{align*}

We understand $C^-$ and $C^+$ as operators acting on functions of variable $\vect{v}_e$. Their weak forms are given by
\begin{align*}
\myint_{R^3} C^- \phi\of{\vect{v}_e} \diff{\vect{v}_e} 
&=
\myint_{R^3} \myint_{R^3} \myint_{S^2} 
B\of{\vect{v}_e, \vect{v}_0, \vect{\omega}} 
f_e\of{\vect{v}_e} f_0\of{\vect{v}_0} 
\phi\of{\vect{v}_e} 
\diff{\vect{v}_e} \diff{\vect{v}_0} \diff{\vect{\omega}}
\\
\myint_{R^3} C^+ \phi\of{\vect{v}_e} \diff{\vect{v}_e} 
&= 
\myint_{R^3} \myint_{R^3} \myint_{S^2} 
B\of{\vect{v}_e^\prime, \vect{v}_0^\prime, \vect{\omega}} 
f_e\of{\vect{v}_e^\prime} f_0\of{\vect{v}_0^\prime} 
\phi\of{\vect{v}_e^\text{post}\of{\vect{v}_e^\prime, \vect{v}_0^\prime, \vect{\omega}}} 
\diff{\vect{v}_0^\prime} \diff{\vect{v}_e^\prime} \diff{\vect{\omega}}
\\
&= 
\myint_{R^3} \myint_{R^3} \myint_{S^2} 
B\of{\vect{v}_e, \vect{v}_0, \vect{\omega}} 
f_e\of{\vect{v}_e} f_0\of{\vect{v}_0} 
\phi\of{\vect{v}_e^\text{post}\of{\vect{v}_e, \vect{v}_0, \vect{\omega}}} 
\diff{\vect{v}_0} \diff{\vect{v}_e} \diff{\vect{\omega}}
\end{align*}
where we integrated out the delta function and renamed dummy variables. Thus the weak form of the total collision operator $C = C^+ - C^-$ can be written as 
\begin{align*}
\myint_{R^3} C \phi\of{\vect{v}_e} \diff{\vect{v}_e} 
&=
\myint_{R^3} \myint_{R^3} \myint_{S^2} 
B\of{\vect{v}_e, \vect{v}_0, \vect{\omega}} 
f_e\of{\vect{v}_e} f_0\of{\vect{v}_0} 
\left(
\phi\of{\vect{v}_e^\text{post}\of{\vect{v}_e, \vect{v}_0, \vect{\omega}}} 
- \phi\of{\vect{v}_e} 
\right)
\diff{\vect{v}_0} \diff{\vect{v}_e} \diff{\vect{\omega}}
\end{align*}

While for our purposes this formulation is all we need, for completeness sake we derive a strong form of the collision operator. To do so, we perform a change of variables in the weak form of gain operator $C^+$ according to:
\begin{align*}
\vect{v}_e^\prime &= \vect{v}_e^\text{pre} \of{\vect{v}_e^{\prime\prime}, \vect{v}_0^{\prime\prime}, \vect{\omega}}
\\
\vect{v}_0^\prime &= \vect{v}_0^\text{pre} \of{\vect{v}_e^{\prime\prime}, \vect{v}_0^{\prime\prime}, \vect{\omega}}
\end{align*}
As result we get
\begin{multline*}
\myint_{R^3} C^+ \phi\of{\vect{v}_e} \diff{\vect{v}_e} 
= 
\myint_{R^3} \myint_{R^3} \myint_{S^2} 
B\of{\vect{v}_e^\text{pre} \of{\vect{v}_e^{\prime\prime}, \vect{v}_0^{\prime\prime}, \vect{\omega}}, \vect{v}_0^\text{pre} \of{\vect{v}_e^{\prime\prime}, \vect{v}_0^{\prime\prime}, \vect{\omega}}, \vect{\omega}} 
\times
\\
\times
f_e\of{\vect{v}_e^\text{pre} \of{\vect{v}_e^{\prime\prime}, \vect{v}_0^{\prime\prime}, \vect{\omega}}} 
f_0\of{\vect{v}_0^\text{pre} \of{\vect{v}_e^{\prime\prime}, \vect{v}_0^{\prime\prime}, \vect{\omega}}} 
\phi\of{\vect{v}_e^{\prime\prime}} 
|J\of{\vect{v}_e^{\prime\prime}, \vect{v}_0^{\prime\prime}, \vect{\omega}}|
\diff{\vect{v}_0^{\prime\prime}} \diff{\vect{v}_e^{\prime\prime}} \diff{\vect{\omega}}
\end{multline*}
or, after renaming dummy variables,
\begin{multline*}
\myint_{R^3} C^+ \phi\of{\vect{v}_e} \diff{\vect{v}_e} 
= 
\myint_{R^3} \myint_{R^3} \myint_{S^2} 
B\of{\vect{v}_e^\text{pre} \of{\vect{v}_e, \vect{v}_0, \vect{\omega}}, \vect{v}_0^\text{pre} \of{\vect{v}_e, \vect{v}_0, \vect{\omega}}, \vect{\omega}} 
\times
\\
\times
f_e\of{\vect{v}_e^\text{pre} \of{\vect{v}_e, \vect{v}_0, \vect{\omega}}} 
f_0\of{\vect{v}_0^\text{pre} \of{\vect{v}_e, \vect{v}_0, \vect{\omega}}} 
\phi\of{\vect{v}_e} 
|J\of{\vect{v}_e, \vect{v}_0, \vect{\omega}}|
\diff{\vect{v}_0} \diff{\vect{v}_e} \diff{\vect{\omega}}
\end{multline*}
where $J\of{\vect{v}_e, \vect{v}_0, \vect{\omega}} = \frac{ \partial \left( \vect{v}_e^\text{pre}\of{\vect{v}_e, \vect{v}_0, \vect{\omega}}, \vect{v}_0^\text{pre}\of{\vect{v}_e, \vect{v}_0, \vect{\omega}} \right)}{\partial \left( \vect{v}_e, \vect{v}_0 \right)}$ is the Jacobian of the transformation of variables. Note that in this last expression $\vect{v}_e$, $\vect{v}_0$ can be interpreted as post-collision velocities. Combining it with the weak form of the loss operator (where $\vect{v}_e$, $\vect{v}_0$ actually stand for pre-collision velocities) we obtain
\begin{multline*}
\myint_{R^3} C \phi\of{\vect{v}_e} \diff{\vect{v}_e} 
=
\myint_{R^3} \myint_{R^3} \myint_{S^2} 
\phi\of{\vect{v}_e} 
\diff{\vect{v}_0} \diff{\vect{v}_e} \diff{\vect{\omega}}
\times
\\
\times
\big(
B^\text{pre}\of{\vect{v}_e, \vect{v}_0, \vect{\omega}}  
f_e^\text{pre}\of{\vect{v}_e, \vect{v}_0, \vect{\omega}}
f_0^\text{pre}\of{\vect{v}_e, \vect{v}_0, \vect{\omega}} 
|J\of{\vect{v}_e, \vect{v}_0, \vect{\omega}}|
\\
-
B\of{\vect{v}_e, \vect{v}_0, \vect{\omega}} 
f_e\of{\vect{v}_e} f_0\of{\vect{v}_0} 
\big)
\end{multline*}
where notation $x^\text{pre}\of{\vect{v}_e, \vect{v}_0, \vect{\omega}} = x\of{\vect{v}_e^\text{pre} \of{\vect{v}_e, \vect{v}_0, \vect{\omega}}, \vect{v}_0^\text{pre} \of{\vect{v}_e, \vect{v}_0, \vect{\omega}}, \vect{\omega}}$ is used. Thus, a strong form of the total collision operator for general binary reactions can be written as
\begin{align*}
C = 
\myint_{R^3} \myint_{S^2} 
\left(
f_e^\text{pre}
f_0^\text{pre}
B^\text{pre} 
|J|
-
f_e
f_0
B
\right)
\diff{\vect{v}_0} \diff{\vect{\omega}}
\end{align*}
In case of elastic collisions it can be shown that $|J| = 1$ and
\begin{align*}
B^\text{pre}\of{\vect{v}_e, \vect{v}_0, \vect{\omega}} &=
B\of{\vect{v}_e^\text{pre} \of{\vect{v}_e, \vect{v}_0, \vect{\omega}}, \vect{v}_0^\text{pre} \of{\vect{v}_e, \vect{v}_0, \vect{\omega}}, \vect{\omega}} 
\\
&=
B\of{|\vect{v}_e^\text{pre}- \vect{v}_0^\text{pre}|, |\left(\vect{v}_e^\text{pre}- \vect{v}_0^\text{pre}\right)\cdot\vect{\omega}|} 
\\
&=
B\of{|\vect{v}_e- \vect{v}_0|, |\left(\vect{v}_e- \vect{v}_0\right)\cdot\vect{\omega}|}
\\
&=
B\of{\vect{v}_e, \vect{v}_0, \vect{\omega}} 
\end{align*}
and the collision operator becomes the familiar
\begin{align*}
C = 
\myint_{R^3} \myint_{S^2} 
\left(
f_e^\text{pre}
f_0^\text{pre}
-
f_e
f_0
\right)
B
\diff{\vect{v}_0} \diff{\vect{\omega}}
\end{align*}

\clearpage
In the derivation above we assumed that the collision kernel $B=B\of{\vect{v}_e, \vect{v}_0, \vect{\omega}}$ is known. However, for electron-heavy particle collisions information is available in terms of collisional cross sections $\sigma = \sigma\of{|\vect{v}_e|, \chi, \theta}$, where $\vect{v}_e$ is the velocity of the incident electron in the frame of reference of the heavy particle, and $\chi$, $\theta$ are angles of scattering. It is usually assumed that collisions occur axisymmetrically, that is $\sigma = \sigma\of{|\vect{v}_e|, \chi}$

\begin{align*}
C^- = \myint_{R^3} \myint_{R^3} \myint_{0}^{\pi} \myint_{0}^{2\pi} 
|\vect{v}_e^\prime-\vect{v}_0^\prime| \sigma\of{|\vect{v}_e^\prime-\vect{v}_0^\prime|, \chi}
f_e\of{\vect{v}_e^\prime} f_0\of{\vect{v}_0^\prime} 
\delta\of{\vect{v}_e^\prime - \vect{v}_e} 
\diff{\vect{v}_0^\prime} \diff{\vect{v}_e^\prime} \sin\of{\chi} \diff{\chi} \diff{\theta}.
\end{align*}
\begin{align*}
C^+ = \myint_{R^3} \myint_{R^3} \myint_{0}^{\pi} \myint_{0}^{2\pi} 
|\vect{v}_e^\prime-\vect{v}_0^\prime| \sigma\of{|\vect{v}_e^\prime-\vect{v}_0^\prime|, \chi}
f_e\of{\vect{v}_e^\prime} f_0\of{\vect{v}_0^\prime} 
\delta\of{\vect{v}_e^\text{post}\of{\vect{v}_e^\prime, \vect{v}_0^\prime, \chi, \theta} - \vect{v}_e} 
\diff{\vect{v}_0^\prime} \diff{\vect{v}_e^\prime} \sin\of{\chi} \diff{\chi} \diff{\theta}.
\end{align*}

\clearpage
\section{Analysis of projection operator}
Consider a distribution function given as an expansion in terms of Maxwell polynomials based on thermal velocity $v_0$
\begin{align*}
f_0\of{v} &= \frac{1}{\left( \sqrt{\pi} v_0 \right)^3} e^{-\left( \frac{v}{v_0} \right)^2} \sum_{j=0}^{N} a_j P_j\of{\frac{v}{v_0}}
\end{align*}
We are interested in obtaining the same distribution function in a different basis corresponding to thermal velocity $v_1$
\begin{align*}
f_1 \of{v} &= \frac{1}{\left( \sqrt{\pi} v_1 \right)^3} e^{-\left( \frac{v}{v_1} \right)^2} \sum_{j=0}^{N} b_j P_j\of{\frac{v}{v_1}}
\end{align*}
A natural way of obtaining coefficients $b_i$ from known coefficients $a_i$ is to ensure that the action of both distribution functions onto a chosen set of test functions coincides
\begin{align*}
\myint_{R} {v}^2 f_0\of{v} \phi\of{v} \diff{v} = 
\myint_{R} {v}^2 f_1\of{v} \phi\of{v} \diff{v}
\end{align*}
Choosing polynomial functions up to order $N$ (more precisely, Maxwell polynomials $P_i\of{\frac{v}{v_1}}$ for sake of convenience) as the set of test functions one gets (using orthogonality)
\begin{align*}
b_i \myint_{R} \left( \frac{v}{v_1} \right)^2 e^{-\left( \frac{v}{v_1} \right)^2} P^2_i\of{\frac{v}{v_1}} \diff{\left(\frac{v}{v_1}\right)} = 
\sum_{j=0}^{N} a_j
\myint_{R} \left( \frac{v}{v_0} \right)^2 e^{-\left( \frac{v}{v_0} \right)^2}  P_j\of{\frac{v}{v_0}} P_i\of{\frac{v}{v_1}} \diff{\left(\frac{v}{v_0}\right)} 
\end{align*}
Note that automatically $b_0 = a_0$, which can be interpreted as mass conservation during projection. 
Let us denote $\vect{a} = \left( a_0,\ldots, a_N \right)^T$ and $\vect{b} = \left( b_0,\ldots, b_N \right)^T$, then the transformation from one expansion to another one can be written as 
\begin{align*}
\vect{b} = \Pi \vect{a}
\end{align*}
where elements of matrix $\Pi$ are given by
\begin{align*}
\Pi_{i,j} &= \myint_{R} \left( \frac{v}{v_0} \right)^2 e^{-\left( \frac{v}{v_0} \right)^2} P_j\of{\frac{v}{v_0}} P_i\of{\frac{v}{v_1}} \diff{\left(\frac{v}{v_0}\right)} 
/ 
\myint_{R} \left( \frac{v}{v_1} \right)^2 e^{-\left( \frac{v}{v_1} \right)^2} P^2_i\of{\frac{v}{v_1}} \diff{\left(\frac{v}{v_1}\right)}
\end{align*}
Assume that the relation between thermal velocities $v_0$ and $v_1$ are given by relative change $\varepsilon$ as
\begin{align*}
v_1 = v_0 \left( 1 + \varepsilon \right)
\end{align*}
Then the expression for $\Pi_{i,j}$ can be written as
\begin{align*}
\Pi_{i,j} = \Pi_{i,j} \of{\varepsilon}
&= \myint_{R} x^2 e^{-x^2} P_i\of{\frac{x}{ 1 + \varepsilon }} P_j\of{x} \diff{x} / \myint_{R} x^2 e^{-x^2} P^2_i\of{x} \diff{x}
\end{align*}

\subsection{Preliminary experimental observations}

Figure~\ref{fig:projection} (a) shows the eiganvalues of the projection operator for a few different values of $\varepsilon$. As one can see, if $\varepsilon > 0$ (projection to a higher temperature) then all eigenvalues seem to be less than one. On the other hand, if $\varepsilon < 0$ (projection to a lower temperature) then all eigenvalues appear to be greater than one, which might indicate of a very unstable behavior. Indeed, all information, including errors, gets amplified during projection. 

\begin{figure}[H]
\label{fig:projection}
\centering
\includegraphics[width=\textwidth]{dat/projection_different_eps.png}
\caption{(a) Eigenvalues of projection operator for different $\varepsilon$; (b) Projection of clean data; (c) Projection of polluted data; (d) Projection of polluted data with small steps and thresholding}
\end{figure}

To inspect the behavior of the projection operator more closely, let us consider projection of a series which coefficients are given by $a_i = \exp\of{-100i}$. The results of projection with different $\varepsilon$ are given in Figure~\ref{fig:projection} (b) and, at the first glance, do not indicate of any instabilities. However, this is likely due to coefficients $a_i$ decaying very fast. Indeed, if we add a random noise to coefficients $a_i$ then the projection results seem to stay stable only if $\varepsilon > 0$ while for $\varepsilon < 0$ high-order coefficients growing uncontrollably (see Figure~\ref{fig:projection} (c)). 

A possible fix could be to use multiple small projection steps with zeroing all coefficients below a chosen threshold after each sub-step. The example of such a strategy is given in  Figure~\ref{fig:projection} (d) where a target relative change $\varepsilon_0 = 0.001$ was used (that is, the number of sub-steps and their size are chosen as $m = \text{ceil}\of{\frac{\log\of{1+\varepsilon}}{\log\of{1+\varepsilon_0}}}$ and $\tilde{\varepsilon} = \sqrt[n]{1+\varepsilon} - 1$). This strategy produces almost identical results as in the case of one full step and does not result in an uncontrollable growth of high-order coefficients for $\varepsilon < 0$. However, it is unclear whether there are any growing instabilities in the low-order part of the spectrum which are not seen due to large values of the coefficients there.


\begin{figure}[H]
\centering
\includegraphics[width=1.2\textwidth]{dat/projection_pos_eigs.png}
\includegraphics[width=1.2\textwidth]{dat/projection_pos_eigs_visual.png}
\caption{Eigenvectors (top) and their visualization (bottom) for $\varepsilon = 0.1$}
\end{figure}

\begin{figure}[H]
\centering
\includegraphics[width=1.2\textwidth]{dat/projection_neg_eigs.png}
\includegraphics[width=1.2\textwidth]{dat/projection_neg_eigs_visual.png}
\caption{Eigenvectors (top) and their visualization (bottom) for $\varepsilon = -0.1$}
\end{figure}


\end{document}
