\documentclass[mathserif, aspectratio=169]{beamer}
\usetheme{odenpecos}
\setbeamertemplate{itemize/enumerate body begin}{\fontsize{8.8}{9}\selectfont}
\setbeamertemplate{itemize/enumerate subbody begin}{\fontsize{7.5}{8}\selectfont}
\setbeamertemplate{itemize/enumerate subsubbody begin}{\fontsize{7.5}{8}\selectfont}

% default search path for figures
\graphicspath{{./../fig/}}

\newcommand{\zapspace}{\topsep=0pt\partopsep=0pt\itemsep=0pt\parskip=0pt}

\usepackage{multicol}
\usepackage{multirow}
\usepackage{pict2e}
%\usepackage{esdiff}
\usepackage{multimedia}
\usepackage{verbatim}
\usepackage{mhchem}
\usepackage{tikz}
\usetikzlibrary{arrows}
\usepackage[percent]{overpic}
\usepackage[absolute,overlay]{textpos}
\usepackage{tikz} % Required for flow chart
\usepackage[caption=false]{subfig}

\newcommand{\overbar}[1]{\mkern 1.5mu\overline{\mkern-1.5mu#1\mkern-1.5mu}\mkern 1.5mu}
\newcommand{\pp}[2]{\frac{\partial #1}{\partial #2}}
\newcommand{\dd}[2]{\frac{d #1}{d #2}}
\newcommand{\DD}[2]{\frac{D #1}{D #2}}
\newcommand{\mm}{\mathbf{minmod}}
\def\etal{{\it et al~}}
\newcommand{\be}{\begin{eqnarray}}
	\newcommand{\ee}{\end{eqnarray}}
\newcommand{\mbb}[1]{\mathbb{#1}} % math blackboard bold
\newcommand{\mcal}[1]{\mathcal{#1}} % math blackboard bold
\newcommand{\mbf}[1]{\mathbf{#1}} % math bold face (for vectors)
\newcommand{\sbf}[1]{\boldsymbol{#1}} % bold face for symbols
\newcommand{\jump}[1]{\llbracket #1 \rrbracket} % jump operator
\newcommand{\avg}[1]{\langle #1 \rangle} % average operator
\newcommand{\rarrow}{\rightarrow}
\newcommand{\Rarrow}{\Rightarrow}
\newcommand{\LRarrow}{\Leftrightarrow}
\newcommand{\vvvert}{|\kern-1pt|\kern-1pt|}
\newcommand{\enorm}[1]{\vvvert #1 \vvvert}
\newcommand{\nutil}{\tilde{\nu}}
\newcommand{\Var}{\mathrm{Var}}
\newcommand{\Cov}{\mathrm{Cov}}


\definecolor{MyDarkGreen}{rgb}{0,0.45,0.08}
\newcommand{\myred}[1]{{\color{red} #1}}
\newcommand{\myblue}[1]{{\color{blue} #1}}
\newcommand{\mygreen}[1]{{\color{MyDarkGreen} #1}}

\newcommand{\sa}{\nu_{\mathrm{sa}}}
\newcommand{\tep}{\tilde{\epsilon}}
\newcommand{\Ssd}{\mathcal{S}} % source term due to slow derivative
\newcommand{\ud}{\,\mathrm{d}}

\newcommand{\Mach}[1]{\ensuremath{\mbox{Ma}_{#1}}}
\newcommand{\Reynolds}{\ensuremath{\mathit{Re}}}
\newcommand{\DensityRat}{\ensuremath{\mathit{DR}}}
\newcommand{\BlowRat}{\ensuremath{\mbox{BR}}}
\newcommand{\VelRat}{\ensuremath{\mathit{VR}}}
\newcommand{\Tau}{\ensuremath{\mathrm{T}}}

\newcommand{\wall}     {\ensuremath{\mathrm{w}}}   % wall subindex
\newcommand{\awall}    {\ensuremath{\mathrm{aw}}}  % adiabatic wall subindex

\newcommand{\commentout}[1]{}

\newcommand{\vect}[1]{\boldsymbol{#1}}
\usepackage{mleftright}
\newcommand{\of}[1]{\mleft( #1 \mright)}
\newcommand{\vth}{v_{\textrm{th}}}
\newcommand{\reals}{\mathbb{R}}
\newcommand{\myint}{\int\limits}
\newcommand{\ddt}[1]{\partial_t #1}
\newcommand{\RR}{\mathbb{R}}
\newcommand{\vr}{v}
\newcommand{\diff}[1]{\, d#1}
\newcommand{\norm}[1]{\left\lVert#1\right\rVert}
%\newcommand{\vtheta}{\theta_{\vect{v}}}
%\newcommand{\vphi}{\varphi_{\vect{v}}}
%\newcommand{\vr}{v_{r}}
\newcommand{\vtheta}{{v_{\theta}}}
\newcommand{\vphi}{v_{\varphi}}
\newcommand{\vomega}{v_{\omega}}
\newcommand{\vrunit}{\hat{\vect{v}}_{r}}
\newcommand{\vthetaunit}{\hat{\vect{v}}_{\theta}}
\newcommand{\vphiunit}{\hat{\vect{v}}_{\varphi}}
\DeclareMathOperator{\variance}{Var}

\usepackage{mathtools}
\DeclarePairedDelimiter\ceil{\lceil}{\rceil}
\DeclarePairedDelimiter\floor{\lfloor}{\rfloor}
\usepackage{tikz}

\usepackage{tcolorbox}
\tcbuselibrary{minted,breakable,xparse,skins}
\definecolor{bg}{gray}{0.95}
\DeclareTCBListing{mintedbox}{O{}m!O{}}{%
	breakable=true,
	listing engine=minted,
	listing only,
	minted language=#2,
	minted style=default,
	minted options={%
		linenos,
		gobble=0,
		breaklines=true,
		breakafter=,
		fontsize=\small,
		numbersep=8pt,
		#1},
	boxsep=0pt,
	left skip=0pt,
	right skip=0pt,
	left=25pt,
	right=0pt,
	top=3pt,
	bottom=3pt,
	arc=5pt,
	leftrule=0pt,
	rightrule=0pt,
	bottomrule=2pt,
	toprule=2pt,
	colback=bg,
	colframe=orange!70,
	enhanced,
	overlay={%
		\begin{tcbclipinterior}
			\fill[orange!20!white] (frame.south west) rectangle ([xshift=20pt]frame.north west);
	\end{tcbclipinterior}},
	#3,
}

\begin{document}
% disable nav
\setbeamertemplate{navigation symbols}{}

% ---------------------------------------------------------------
% Oden/Pecos title page

\hoffset=.16in

\begin{frame}[plain,t]{}
\makeatletter
%\vspace*{0.85cm}
%\vspace*{0.65cm}
\includegraphics[height=0.9in,trim=50 40 40 0, clip]{PMSc_159_university_formal_horizontal.pdf} \newline
%\vspace*{0.3cm}
\begin{columns}[T,onlytextwidth]
\column{.8\textwidth}
{\bf \color{burntorange} \fontfamily{bch}\selectfont 
% -- Set talk title here
Integration efforts on the electron Boltzmann solver with the torch plasma simulator
%Electron Boltzmann solver integration with the torch plasma simulator
% --
}
\end{columns}
\vspace*{.15cm}
\rule{.8\textwidth}{0.6pt} \newline

\vspace*{0.05cm}
\setstretch{0.65}
{\fontfamily{phv}\selectfont
  { \scriptsize
    % -- define presenter, authors here
    Milinda Fernando, Kinetic solvers, Parla, and TPS team members \\
    % --
  }
  {\color{burntorange} \tiny
    % -- define role, meeting event, location, etc
    PSAAP III Annual Review $\cdot$ November 08-09, 2023
    %PSAAP TST Meeting $\cdot$ April 24-25, 2023
    % --
  }
}

\vspace*{1cm}
%\includegraphics[height=0.3in]{figures/pecos_orange1.png}
\begin{columns}
\begin{column}{0.8\linewidth}
\includegraphics[height=0.5in]{oden_pecos_2020_wordmark.png}\\
{\scriptsize \url{https://pecos.oden.utexas.edu}}
\end{column}

\begin{column}{0.2\linewidth}
\includegraphics[height=0.6in]{psaap3-logo.png}
\end{column}
\end{columns}

\end{frame}
\hoffset=0in
% -- end title slide ---------------------------------------------

%\begin{frame}
%	\frametitle{Outline}
%	\begin{itemize}
%		\item Spatially homogeneous Boltzmann equation, $\partial_t f - \frac{\vect{E} q}{m} \cdot \nabla_{\vect{v }}f = C(f)$
%		\item Representation of $f$ (i.e., isotropic + anisotropic correction terms), use spherical harmonics for angular directions + experimentation of basis functions in radial direction. 
%		\begin{itemize}
%			\item Global approximations with Maxwell and Laguerre polynomials.
%			\item Local approximations with linear and higher order B-splines
%		\end{itemize}
%		\item Collision operator (5d integral form)
%		\item Simplifications for the collision operator with analytical integration of angular directions (1d integral form). 
%		\item Equations for the steady-state solution (spatially homogeneous case) 
%		\item Two-term formulation vs. EEDF formulation with diffusion term. 
%		\item Verification with Bolsig+ code.  (Both approaches, importance of the diffusion term), PS 2.4
%		\item Formulation for 1D-space+3D-velocity space Boltzmann equations with some preliminary results for 1d glow discharge problem. 
%		\item Discuss on single GPU implementation with CuPy, Challenges in 1D+3V formulation (i.e., boundary conditions) and Future work, ES 2.6
%	\end{itemize}
%\end{frame}

\begin{frame}
	\frametitle{Where do we fit ?}
	\begin{figure}
		\centering
		\includegraphics[width=0.9\textwidth]{where_we_fit.png}
	\end{figure}
\end{frame}

%\begin{frame}
%	\frametitle{Outline}
%\end{frame}

\begin{frame}
	\frametitle{Torch plasma simulator (TPS)}
	\begin{columns}
		\begin{column}{0.48\textwidth}
			\textbf{TPS with LTE}
			\vspace{0.25in}
			\footnotesize
			\begin{align*}
				\partial_t n_i + \nabla_{\vect{x}} \cdot \vect{J_{n_i}}  = k_i n_0 n_i  \text{ in } \Omega_x \times (0,T]\\
				\partial_t n_0 + \nabla_{\vect{x}} \cdot \vect{J_{n_0}}  = -k_i n_0 n_i \text{ in } \Omega_x \times (0,T]\\
				\text{conservation of momentum for }  Ar^{+}, Ar  \\
				\text{ additional equations for } \vect{E}, T_g , \\
			\end{align*}
			\begin{itemize}
				\item Assumes local thermodynamic equilibrium, (i.e., $T_e = T_g$) and quasi-neutrality (i.e., $n_e=n_i$)
				\item $k_i \approx M(T_g)$ Assumes, electron EEDF is Maxwellian
			\end{itemize}
		\end{column}
		\begin{column}{0.48\textwidth}
			\textbf{Fully coupled TPS + Boltzmann}
			\vspace{0.25in}
			\footnotesize
			\begin{align*}
				&\partial_t n_i + \nabla_{\vect{x}} \cdot \vect{J_{n_i}}  = k_i n_0 n_i  \text{ in } \Omega_x \times (0,T]\\
				&\partial_t n_0 + \nabla_{\vect{x}} \cdot \vect{J_{n_0}}  = -k_i n_0 n_i \text{ in } \Omega_x \times (0,T] \\
				&\text{conservation of momentum for }  Ar^{+}, Ar \\
				&\text{ additional equations for } \vect{E}, T_g \\
				&\partial_t f  + \vect{v} \cdot \nabla_{\vect{x}} f -\frac{\vect{E} q}{m} \cdot \nabla_{\vect{v }}f = C(f, n_0, T_g) \text{ in } \Omega_x \times \mathcal{R}^3 \times (0,T]
			\end{align*}
			\begin{itemize}
				\item $T_e \sim \int_{\vect{v}} \norm{\vec{v}}^2 \hat{f} \diff{\vect{v}}$
				\item $k_i = \int_{\vect{v}} \sigma(\norm{\vect{v}}) \hat{f} \diff{\vect{v}}$ where $\hat{f} = \frac{f}{\int_{\vect{v}} f \diff{\vect{v}}}$
				\item 7D problem
			\end{itemize}
		\end{column}
	\end{columns}

\end{frame}

\begin{frame}
	\frametitle{TPS + Boltzmann}
	% \begin{itemize}
	% 	\item Importance of spatially coupled Boltzmann on torch QoIs to be determined
	% 	\item Current approach: Batch of spatially homogeneous Boltzmann solves for the $\Omega_x$
	% \end{itemize}
	\vspace{0.2in}
	\begin{columns}
		\begin{column}{0.48\textwidth}
			\textbf{Strong coupling (2-way)}\\
			\begin{figure}
				\resizebox{0.9\textwidth}{!}{
					\centering
					\begin{tikzpicture}[block/.style={draw=black,thick, inner sep=2pt, rounded corners, minimum width=2cm, minimum height=1.2cm, fill=rightfooterorange,font={\small}}, shift=(current page.center)]
						\begin{scope}[yshift=10cm,xshift=5cm]
							% \draw[lightgray]
							% (current page.north) -- (current page.south)
							% (current page.west)  -- (current page.east);
							\node[block] (A) at (-9, 3) {TPS};
							\node[block] (C) at (-3 , 3.001) {Boltzmann};
							% \node[block] (D) at (0.002 , 0) {electron kinetic coefficients ($T_g, E/n_0, n_e/n_0, E$)};
							% \draw [->,very thick] (B) -- node [text width=3cm, midway, above, align=center,xshift =-0.8cm] {with Maxwellian assumption $f = M_{T_e}(\vect{v})$} (A);\pause
							% \draw [->,very thick] (B) -- node [text width=3cm, midway, above, align=center,xshift = 1.6cm, yshift=-0.5cm] {collision operator assembly} (C);
							% \draw [->,very thick] (C) -- node [text width=3cm, midway, above, align=center, yshift=-0.5cm, xshift=1.2cm] {static E-field assumption} (D);
							% \draw [->,very thick] (D) -- node [text width=3cm, midway, above, align=center,xshift =-1.2cm, yshift=-1.8cm] {$T_e$-based \\ interpolation for rates, kinetic coefficients} (A);\pause
							\draw [->,very thick] (A) to[bend left  = 8] node [text width=3cm, midway, above, align=center,yshift=0cm] {$T_g, E(x,t), n_e, n_0$} (C);
							\draw [->,very thick] (C) to[bend left  = 8] node [text width=3cm, midway, below, align=center,xshift=0.0cm] {$k_i$, $n_e$, $T_e$} (A);
						\end{scope}
				\end{tikzpicture}}
			\end{figure}
		\end{column}
		\begin{column}{0.48\textwidth}
			\textbf{Weak coupling (1-way)}\\
			\begin{figure}
				\resizebox{0.9\textwidth}{!}{
					\centering
					\begin{tikzpicture}[block/.style={draw=black,thick, inner sep=2pt, rounded corners, minimum width=2cm, minimum height=1.2cm, fill=rightfooterorange,font={\small}}, shift=(current page.center)]
						\begin{scope}[yshift=10cm,xshift=5cm]
							% \draw[lightgray]
							% (current page.north) -- (current page.south)
							% (current page.west)  -- (current page.east);
							\node[block] (A) at (-9, 3) {TPS};
							\node[block] (C) at (-3 , 3.001) {Boltzmann};
							\node[] (B) at (-9, 1.2) {$k_i^{tps}$};
							\node[] (D) at (-3, 1.2) {$k_i^{bte}$};

							% \node[block] (D) at (0.002 , 0) {electron kinetic coefficients ($T_g, E/n_0, n_e/n_0, E$)};
							% \draw [->,very thick] (B) -- node [text width=3cm, midway, above, align=center,xshift =-0.8cm] {with Maxwellian assumption $f = M_{T_e}(\vect{v})$} (A);\pause
							% \draw [->,very thick] (B) -- node [text width=3cm, midway, above, align=center,xshift = 1.6cm, yshift=-0.5cm] {collision operator assembly} (C);
							% \draw [->,very thick] (C) -- node [text width=3cm, midway, above, align=center, yshift=-0.5cm, xshift=1.2cm] {static E-field assumption} (D);
							% \draw [->,very thick] (D) -- node [text width=3cm, midway, above, align=center,xshift =-1.2cm, yshift=-1.8cm] {$T_e$-based \\ interpolation for rates, kinetic coefficients} (A);\pause
							\draw [->,very thick] (A) to[bend left  = 8] node [text width=3cm, midway, above, align=center,yshift=0cm] {$T_g, E(x,t), n_e, n_0$} (C);

							\draw [->,very thick] (A) to node [text width=3cm, midway, above, align=center,yshift=0cm] {} (B);
							\draw [->,very thick] (C) to node [text width=3cm, midway, above, align=center,yshift=0cm] {} (D);
							\draw [<->,very thick] (B) to node [text width=3cm, midway, below, align=center,xshift=0.0cm] {compare} (D);
						\end{scope}
				\end{tikzpicture}}
			\end{figure}
		\end{column}
	\end{columns}
	Current integration
	\begin{itemize}
		\item Focus on weak coupling to evaluate modeling errors
		\item Take, $n_0, T_g, n_i, \vect{E}$ from TPS code, and compare $k_i\approx$ Maxwellian EEDF to more generic electron model (i.e., Boltzmann equation) 
	\end{itemize}
\end{frame}

\begin{frame}
	\frametitle{Current Integration}
	\begin{itemize}
		\item \textbf{A} : Assume LTE $T_e=T_g$ and $f_M(\vect{v}, t)=M_{T_e}(\vect{v})$ and $k_i^{M} = \mathcal{K}f_M$
		\item \textbf{B} : Batched 0D-Boltzmann solver with static $\vect{E}$ , $k_i^{B} = \mathcal{K}f$ with weak-coupling
		\item \textbf{C} : Batched 0D-Boltzmann solver with oscillatory $\vect{E}$ field with cycle averaged rates, $k_i^{C} = \mathcal{K}f$ with weak-coupling
		\item In-terms of accuracy, $\text{\textbf{A}} < \text{\textbf{B}} < \text{\textbf{C}} < \text{Fully coupled TPS + Boltzmann}$
	\end{itemize}
	%We compare \textbf{B} vs. \textbf{A}, \textbf{C} vs. \textbf{A} and \textbf{B} vs. \textbf{C}
\end{frame}

\begin{frame}
	\frametitle{A Vs. B}
	\begin{itemize}
		\item \textbf{A} : Assume LTE $T_e=T_g$ and $f_M(\vect{v}, t)=M_{T_e}(\vect{v})$ and $k_i^{M} = \mathcal{K}f_M$
		\item \textbf{B} : Batched 0D-Boltzmann solver with static $\vect{E}$ , $k_i^{B} = \mathcal{K}f$ with weak-coupling
		%\item \textbf{C} : Batched 0D-Boltzmann solver with oscillatory $\vect{E}$ field with cycle averaged rates, $k_i^{C} = \mathcal{K}f$ with weak-coupling
	\end{itemize}

	\begin{figure}
		\begin{center}
			\hspace{-2.25in}
			\includegraphics[width=1.3\textwidth]{tps_0d2v_ss.png}
		\end{center}
	\end{figure}
\end{frame}

\begin{frame}
	\frametitle{A Vs. C}
	\begin{itemize}
		\item \textbf{A} : Assume LTE $T_e=T_g$ and $f_M(\vect{v}, t)=M_{T_e}(\vect{v})$ and $k_i^{M} = \mathcal{K}f_M$
		%\item \textbf{B} : Batched 0D-Boltzmann solver with static $\vect{E}$ , $k_i^{B} = \mathcal{K}f$ with weak-coupling
		\item \textbf{C} : Batched 0D-Boltzmann solver with oscillatory $\vect{E}$ field with cycle averaged rates, $k_i^{C} = \mathcal{K}f$ with weak-coupling
	\end{itemize}
	\begin{figure}
		\begin{center}
			\hspace{-2.25in}
			\includegraphics[width=1.3\textwidth]{tps_0d2v_ts.png}
		\end{center}
	\end{figure}
\end{frame}

\begin{frame}
	\frametitle{B Vs. C}
	\begin{itemize}
		%\item \textbf{A} : Assume LTE $T_e=T_g$ and $f_M(\vect{v}, t)=M_{T_e}(\vect{v})$ and $k_i^{M} = \mathcal{K}f_M$
		\item \textbf{B} : Batched 0D-Boltzmann solver with static $\vect{E}$ , $k_i^{B} = \mathcal{K}f$ with weak-coupling
		\item \textbf{C} : Batched 0D-Boltzmann solver with oscillatory $\vect{E}$ field with cycle averaged rates, $k_i^{C} = \mathcal{K}f$ with weak-coupling
	\end{itemize}
	\begin{figure}
		\begin{center}
			%\hspace{-2.25in}
			%\vspace{-0.5in}
			\includegraphics[width=0.8\textwidth]{tps_0d2v_ts_vs_ss.png}
		\end{center}
	\end{figure}
\end{frame}


% \begin{frame}
% 	\frametitle{Boltzmann equation}
% 	\begin{itemize}
% 		\item \textbf{Objective} : Enable accurate plasma simulations by solving the Boltzmann equation for electron distribution function.
% 		\item For a given electric field $\vect{E}$
% 		\begin{align}
% 			\partial_t f -\frac{\vect{E} q}{m} \cdot \nabla_{\vect{v }}f = C_{en}f + C_{ee}(f)
% 		\end{align}
% 		\item $C_{en}$ : electron-heavy collisions (i.e., elastic, ionization)
% 		\begin{itemize}
% 			\item Use LXCAT cross-section data 
% 			\item Linear in $f$
% 		\end{itemize} 
% 		\item $C_{ee}$ : electron-electron Columbic collisions 
% 		\begin{itemize}
% 			\item Modeled using Fokker-Plank equation with analytical cross-section
% 			\item Nonlinear in $f$
% 		\end{itemize}
% 		\item With $f$ we can compute electron kinetic coefficients, reaction rates and other QoIs. 
% 		%\item $C_{ei}$: electron-ion Columbic can be modeled similarly if needed
% 		%		\item Derived the steady state equation
% 		%		\begin{align}
% 			%			\textcolor{black!70}{\partial_t \hat{f} = -(u^T C \hat{f}) \hat{f} + (C+E)\hat{f} \text{ where } \hat{f}(\vect{v},t) = \frac{f(v,t)}{\myint_{R^3} f(\vect{v},t) \diff{\vect{v}}}}\\
% 			%			\textcolor{black!70}{\partial_t (\hat{f}) = 0 \ \ \  \Rightarrow} \ \ \  -(u^T C \hat{f}) \hat{f} + (C+E)\hat{f} =0  \text{ with } u^T \hat{f}-1=0
% 			%		\end{align}
% 	\end{itemize}
% \end{frame}

% \begin{frame}
% 	\frametitle{Discretization}
% 	\small
% 	\begin{itemize}
% 		\item Representation of $f\of{\vect{v},t} = \sum_{klm} f_{klm} \underbrace{\phi_k\of{v}}_{\text{B-Spline basis}} \underbrace{Y_{lm}\of{v_\theta, v_\phi}}_{\tiny\text{sph. harm.}}$ 
% %		\item Weak formulation
% %		$
% %		\displaystyle
% %		\quad
% %		\partial_t f - \frac{\vect{E} q}{m} \cdot \nabla_{\vect{v}}f = C(f)
% %		\quad $ \\
% %		$
% %		\displaystyle
% %		\quad
% %		\Rightarrow \quad
% %		\partial_t \myint_{R^3} f \phi\of{\vect{v}} \ud \vect{v} = 
% %		\myint_{R^3} C(f) \phi\of{\vect{v}} \ud \vect{v} + \myint_{R^3} \of{\frac{\vect{E} q}{m} \cdot \nabla_{\vect{v}} f} \phi(\vect{v}) \ud \vect{v}\text{ , } 
% %		\forall \phi(\vect{v})$
% 		\item Discretized system $\Rightarrow$
% 		$
% 		\displaystyle
% 		\quad
% 		\partial_t f  = C_{en}f + C_{ee}(f,f) + E A_v f
% 		$ where $C_{en}, A_v$ are matrices and $C_{ee}$ is a rank 3 tensor\\
% 		%\item In spatially homogeneous case (with isotropic scattering) azimuthal symmetry is preserved (i.e., 2 dimensions in v-space)
% 		\item Most cases can be resolved with 128 splines in radial direction with 2-spherical modes $\Rightarrow$ 256 DoFs in total
% 		%		\item Weak form of the collision operator (5d integral) \\
% 		%		$
% 		%		\displaystyle
% 		%		\quad 
% 		%		\myint_{R^3} C_{en} \phi\of{\vect{v}_e} \diff{\vect{v}_e} 
% 		%		=
% 		%		N \myint_{R^3} \myint_{S^2} 
% 		%		v\sigma(v,\vect{\omega})
% 		%		f_e\of{\vect{v}_e}
% 		%		\left(
% 		%		\psi\of{\vect{v}_e^\text{post}\of{\vect{v}_e, \vect{\omega}}} 
% 		%		- \psi\of{\vect{v}_e} 
% 		%		\right)
% 		%		\diff{\vect{v}_e} \diff{\vect{\omega}}
% 		%		$
% 		%		\item Isotropic scattering, azimuthal symmetry, and using spherical harmonics addition (1d integral)\\
% 		%		$
% 		%		\displaystyle
% 		%		\quad 
% 		%		C_{en}^{ql} 
% 		%		=
% 		%		N \myint_{0}^{\infty} 
% 		%		v^3\sigma(v)\delta_{ql}
% 		%		f_e^{l}\of{v}
% 		%		\left(
% 		%		\delta_{q0}\psi\of{v_e^\text{post}\of{v}} - \psi\of{v} 
% 		%		\right)
% 		%		\diff{v} 
% 		%		$ 
% 	\end{itemize}
% \end{frame}

% \begin{frame}
% 	\frametitle{Overview: TPS + 0D2V Boltzmann coupling}
% 	\begin{figure}
% 		\resizebox{0.9\textwidth}{!}{
% 			\centering
% 			\begin{tikzpicture}[block/.style={draw=black,thick, inner sep=2pt, rounded corners, minimum width=2cm, minimum height=1.2cm, fill=rightfooterorange,font={\small}}, shift=(current page.center)]
% 				\begin{scope}[yshift=10cm,xshift=5cm]
% 					\draw[lightgray]
% 					(current page.north) -- (current page.south)
% 					(current page.west)  -- (current page.east);
% 					\node[block] (A) at (-9, 3) {torch plasma simulator};
% 					\node[block] (B) at (0 , 6) {cross-section data};
% 					\node[block] (C) at (0.001 , 3.001) {0D2V-Boltzmann};
% 					\node[block] (D) at (0.002 , 0) {electron kinetic coefficients ($T_g, E/n_0, n_e/n_0, E$)};
% 					\draw [->,very thick] (B) -- node [text width=3cm, midway, above, align=center,xshift =-0.8cm] {with Maxwellian assumption $f = M_{T_e}(\vect{v})$} (A);\pause
% 					\draw [->,very thick] (B) -- node [text width=3cm, midway, above, align=center,xshift = 1.6cm, yshift=-0.5cm] {collision operator assembly} (C);
% 					\draw [->,very thick] (C) -- node [text width=3cm, midway, above, align=center, yshift=-0.5cm, xshift=1.2cm] {static E-field assumption} (D);
% 					\draw [->,very thick] (D) -- node [text width=3cm, midway, above, align=center,xshift =-1.2cm, yshift=-1.8cm] {$T_e$-based \\ interpolation for rates, kinetic coefficients} (A);\pause
% 					\draw [->,very thick,red] (A) to[bend left  = 8] node [text width=3cm, midway, above, align=center,yshift=0cm] {$T_g, E(x,t), n_e, n_0$} (C);
% 					\draw [->,very thick,red] (C) to[bend left  = 8] node [text width=3cm, midway, below, align=center,xshift=1.0cm] {rate coefficients, mobility, diffusion} (A);
% 				\end{scope}
% 		\end{tikzpicture}}
% 	\end{figure}
% \end{frame}

% \begin{frame}
% 	\frametitle{Methodology: TPS + 0D2V Boltzmann}
% 	\begin{itemize}
% 		\item Launch independent BTE solves for each spatial point roughly 256 DoF per grid point
% 		\item Challenges
% 		\begin{itemize}
% 			\item \textbf{computational cost}: Expensive to launch BTE solves for each spatial DoF 
% 			\item \textbf{grid truncation}: $v_{max} = \gamma \sqrt{\epsilon_{max}}$ v-space grid needs to be truncated depending on BTE input parameters
% 			\item \textbf{memory footprint}: Infeasible to have independent v-space grid for each spatial point
% 		\end{itemize}
% 		\item \textbf{p-coarsening}: Launch BTE solve for each TPS mesh cell 
% 		\item \textbf{Memory footprint}: Spatial points are clustered based on the approximate electron temperature ($T_e$), number of clusters can be specified by the user. Each cluster of points have the same v-space grid.
% 		\item \textbf{Grid truncation} : $T_e$ is also used to determine $v_{max}$
% 		\item For the TPS $T_e \approx$  0.02 eV to 0.8 eV
		
		
% %		Launch 0D2V BTE solves for each spatial gird point
% %		\item BTE require ~ 256 DoF per grid point
% %		\item One-way coupling
% %		\begin{itemize}
% %			\item Evolve the TPS solution till time-periodic steady-state
% %		\end{itemize}
% 	\end{itemize}
% \end{frame}

% \begin{frame}
% 	\frametitle{0D2V batched BTE solver}
% 	\begin{columns}
% 		\begin{column}{0.4\textwidth}
% 			\includegraphics[width=\columnwidth]{tps_input0.png}
% 		\end{column}
% 		\begin{column}{0.6\textwidth}
% 			For each cluster A
% 			\begin{itemize}
% 				\item Let $N_A$ be the number of spatial points
% 				\item Right-hand-side computation done by stacking BTE DoFs 
% 				\begin{align*}
% 					\underbrace{C}_{\text{v-space operator for cluster A}} \underbrace{\begin{bmatrix}
% 						\vdots & \vdots & \hdots &\vdots\\
% 						f_1    & f_2    & \hdots & f_{N_A} \\
% 						\vdots & \vdots & \hdots &\vdots
% 					\end{bmatrix}}_{\text{batched DoFs for cluster A}}
% 				\end{align*}
% 				\item \texttt{Numpy} and \texttt{Cupy} is used for efficient linear algebra computations in CPUs and GPUs
% 			\end{itemize}
% 		\end{column}
% 	\end{columns}
% \end{frame}

\begin{frame}
	\frametitle{Key software components}
	\begin{itemize}
		\item \textbf{Plasma torch simulator (TPS)}: MFEM based discontinuous Galerkin discretization for the torch
		\begin{itemize}
			\item CPU, GPU support through MFEM library
			\item Python interface
		\end{itemize}
		\item \textbf{0D-space Boltzmann solver}
		\begin{itemize}
			\item Galerkin discretization for the v-space
			\item Verified with Bolsig+ and PIC-DSMC codes
			\item Direct steady-state solution and fully-implicit time integrators for transient solutions
			\item Python implementation
			\item CPU \& GPU support through \texttt{numpy} and \texttt{cupy}
		\end{itemize}
		\item \textbf{Parla}: Generic framework for task-based parallelism in Python
			\begin{itemize}
			\item Ability to handle fine grain tasks
			\item Support for CPU/GPU tasks with asynchronous kernel execution
			\item C/C++ based underlying task scheduler
			\end{itemize}
	\end{itemize}
	Clusters can be processed independently
	\begin{itemize}
		\item \textbf{Parla} is used for parallelization of clusters between GPUs on the same node
		\item \textbf{MPI} is used for distributed memory parallelism
	\end{itemize}
\end{frame}





\begin{frame}[fragile]
\frametitle{Example: TPS + 0D2V Boltzmann coupling}
\begin{mintedbox}{python}%[break at=.8\textheight]
import libtps
from   bte_0d3v_batched import bte_0d3v_batched as BoltzmannSolver
comm = MPI.COMM_WORLD
tps = libtps.Tps(comm)
boltzmann = Boltzmann0D2VBactchedSolver(tps, comm)
interface = libtps.Tps2Boltzmann(tps)
it = 0
while it < max_iters:
    tps.solveStep()
    tps.push(interface)
    boltzmann.fetch(interface)
    boltzmann.solve()
    boltzmann.push(interface)
    tps.fetch(interface)
    it = it+1
\end{mintedbox}
\end{frame}

\begin{frame}
	\frametitle{TPS + 0D2V Boltzmann weak coupling}
	\vspace{-0.2in}
	\begin{enumerate}
		\item Let $\vect{U}_0$ be the initial state for TPS (3-species, axisymmetric case)
		\item Evolve TPS solution until time-periodic $\vect{U}_{s}$ solution
		\item Extract TPS fields from $\vect{U}_s$ and start BTE solve with $E(\vect{x}, t) = E_x \cos(\omega t) + E_y \sin(\omega t)$, with Maxwellian initial conditions ($\vect{f}_0$)
		\item Evolve BTE solution $\vect{f}_0$ until time-periodic solutions $\vect{f}_s$
		\item Compute cycle-averaged reaction rates from $\vect{f}_s$ and compare with the rates coefficients used in the TPS code
	\end{enumerate}
\pause
\textbf{Problem setup}
\begin{columns}
	\begin{column}{0.35\textwidth}

		\begin{itemize}
			\item TPS mesh elements 7598
			\item Number of spatial clusters 4
			\item DoFs per spatial point 256, i.e.,  2M DoFs total
			\item Number of timesteps 4000
			\item Number of GPUs 2 Nvidia A100
		\end{itemize}		
	\end{column}
	\begin{column}{0.7\textwidth}
		\begin{itemize}
			\item BTE grid setup time $\approx$ 29.3 s
			\item Assuming static E-field (direct steady-state solve) $\approx$ 8.07 s
			\item Cycle-averaged solution with time-integration (2 cycles, 4000 timesteps) $\approx$ 1314.0 s
		\end{itemize}
	\end{column}
\end{columns}

\end{frame}



% \begin{frame}
% 	\frametitle{TPS + 0D2V Boltzmann weak coupling}
% 	\begin{figure}
% 		\centering
% 		\includegraphics[width=0.5\textwidth]{initial_boltzmann_rxn_comparison.png}
% 	\end{figure}
% \end{frame}

% \begin{frame}
% 	\frametitle{TPS + 0D2V Boltzmann weak coupling}
% 	\vspace{-0.3in}
% 	\begin{figure}
% 		\centering
% 		\includegraphics[width=0.9\textwidth]{tps_bte_comparison1.png}
% 	\end{figure}
% \end{frame}
%\begin{frame}
%	\frametitle{0D-Space Boltzmann one-way coupling}
%	
%\end{frame}

\begin{frame}
	\frametitle{Future work}
	\begin{itemize}
		\item Characterize sensitivity for electron kinetic coefficients to torch stability profiles
		\item Initial work on the strong coupling, performance improvements for the batched solver
		\item 1D2V Boltzmann initial developments for the glow discharge problem
	\end{itemize}
	\begin{center}
		Thank You!
	\end{center}
\end{frame}
\end{document}
