%\documentclass[landscape,a0paper,fontscale=0.285]{baposter} % a0 is 841mm x 1189mm
\documentclass[landscape,archE,fontscale=0.285]{baposter} % archE is 36in x 48in

% Define colors from UT web page
\selectcolormodel{RGB}
%\definecolor{utburntorange}{cmyk}{0,0.2549,0.3922,0.03529}
\definecolor{burntorange}{RGB}{191,87,0}
\definecolor{rightfooterorange}{RGB}{248,151,31}
\definecolor{pms432}{RGB}{51,63,72}
\definecolor{pms7469}{RGB}{0,95,134}
\definecolor{pms7572}{RGB}{214,210,196}
\definecolor{pms7543}{RGB}{156,173,183}


\usepackage{graphicx} % Required for including images
\graphicspath{{../fig}}

\usepackage{graphbox}
\usepackage{overpic}

\usepackage{hyperref}
\hypersetup{
  colorlinks=true,
  urlcolor=pms432,
  pdftitle={Pecos Poster Template},
}

\usepackage{listings}
\lstset{numbers=left,
  basicstyle=\tiny\ttfamily,
  keywordstyle=\color{blue},
  breaklines=true,
  showtabs=false,
  showstringspaces=false,
  xleftmargin=2.5em,
}

\usepackage{amsmath} % For typesetting math
\usepackage{amssymb} % Adds new symbols to be used in math mode

\usepackage{booktabs} % Top and bottom rules for tables
\usepackage{enumitem} % Used to reduce itemize/enumerate spacing
\usepackage{palatino} % Use the Palatino font
\usepackage[font=small,labelfont=bf]{caption} % Required for specifying captions to tables and figures

\usepackage{multicol} % Required for multiple columns
\setlength{\columnsep}{1.5em} % Slightly increase the space between columns
\setlength{\columnseprule}{0mm} % No horizontal rule between columns

\usepackage{tikz} % Required for flow chart
\usetikzlibrary{shapes,arrows} % Tikz libraries required for the flow chart in the template

\newcommand{\compresslist}{ % Define a command to reduce spacing within itemize/enumerate environments, this is used right after \begin{itemize} or \begin{enumerate}
\setlength{\itemsep}{1pt}
\setlength{\parskip}{0pt}
\setlength{\parsep}{0pt}
}
\newcommand{\overbar}[1]{\mkern 1.5mu\overline{\mkern-1.5mu#1\mkern-1.5mu}\mkern 1.5mu}
\newcommand{\pp}[2]{\frac{\partial #1}{\partial #2}}
\newcommand{\dd}[2]{\frac{d #1}{d #2}}
\newcommand{\DD}[2]{\frac{D #1}{D #2}}
\newcommand{\mm}{\mathbf{minmod}}
\def\etal{{\it et al~}}
\newcommand{\be}{\begin{eqnarray}}
	\newcommand{\ee}{\end{eqnarray}}
\newcommand{\mbb}[1]{\mathbb{#1}} % math blackboard bold
\newcommand{\mcal}[1]{\mathcal{#1}} % math blackboard bold
\newcommand{\mbf}[1]{\mathbf{#1}} % math bold face (for vectors)
\newcommand{\sbf}[1]{\boldsymbol{#1}} % bold face for symbols
\newcommand{\jump}[1]{\llbracket #1 \rrbracket} % jump operator
\newcommand{\avg}[1]{\langle #1 \rangle} % average operator
\newcommand{\rarrow}{\rightarrow}
\newcommand{\Rarrow}{\Rightarrow}
\newcommand{\LRarrow}{\Leftrightarrow}
\newcommand{\vvvert}{|\kern-1pt|\kern-1pt|}
\newcommand{\enorm}[1]{\vvvert #1 \vvvert}
\newcommand{\nutil}{\tilde{\nu}}
\newcommand{\Var}{\mathrm{Var}}
\newcommand{\Cov}{\mathrm{Cov}}


\definecolor{MyDarkGreen}{rgb}{0,0.45,0.08}
\newcommand{\myred}[1]{{\color{red} #1}}
\newcommand{\myblue}[1]{{\color{blue} #1}}
\newcommand{\mygreen}[1]{{\color{MyDarkGreen} #1}}

\newcommand{\sa}{\nu_{\mathrm{sa}}}
\newcommand{\tep}{\tilde{\epsilon}}
\newcommand{\Ssd}{\mathcal{S}} % source term due to slow derivative
\newcommand{\ud}{\,\mathrm{d}}

\newcommand{\Mach}[1]{\ensuremath{\mbox{Ma}_{#1}}}
\newcommand{\Reynolds}{\ensuremath{\mathit{Re}}}
\newcommand{\DensityRat}{\ensuremath{\mathit{DR}}}
\newcommand{\BlowRat}{\ensuremath{\mbox{BR}}}
\newcommand{\VelRat}{\ensuremath{\mathit{VR}}}
\newcommand{\Tau}{\ensuremath{\mathrm{T}}}

\newcommand{\wall}     {\ensuremath{\mathrm{w}}}   % wall subindex
\newcommand{\awall}    {\ensuremath{\mathrm{aw}}}  % adiabatic wall subindex

\newcommand{\commentout}[1]{}

\newcommand{\vect}[1]{\boldsymbol{#1}}
\usepackage{mleftright}
\newcommand{\of}[1]{\mleft( #1 \mright)}
\newcommand{\vth}{v_{\textrm{th}}}
\newcommand{\reals}{\mathbb{R}}
\newcommand{\myint}{\int\limits}
\newcommand{\ddt}[1]{\partial_t #1}
\newcommand{\RR}{\mathbb{R}}
\newcommand{\vr}{v}
\newcommand{\diff}[1]{\, d#1}
\newcommand{\norm}[1]{\left\lVert#1\right\rVert}
%\newcommand{\vtheta}{\theta_{\vect{v}}}
%\newcommand{\vphi}{\varphi_{\vect{v}}}
%\newcommand{\vr}{v_{r}}
\newcommand{\vtheta}{v_{\theta}}
\newcommand{\vphi}{v_{\varphi}}
\newcommand{\vomega}{v_{\omega}}
\newcommand{\vrunit}{\hat{\vect{v}}_{r}}
\newcommand{\vthetaunit}{\hat{\vect{v}}_{\theta}}
\newcommand{\vphiunit}{\hat{\vect{v}}_{\varphi}}
\DeclareMathOperator{\variance}{Var}


\begin{document}

\begin{poster}
{
columns=4, % number of columns (change to suite your needs)
headerborder=closed, % Adds a border around the header of content boxes
colspacing=1em, % Column spacing
bgColorOne=white, % Background color for the gradient on the left side of the poster
bgColorTwo=white, % Background color for the gradient on the right side of the poster
borderColor=burntorange, % Border color
headershade=plain,
headerColorOne=burntorange, % Background color for the header in the content boxes (left side)
headerColorTwo=burntorange, %burntorange, % Background color for the header in the content boxes (right side)
headerFontColor=white, % Text color for the header text in the content boxes
boxColorOne=white, % Background color of the content boxes
textborder=roundedleft, % Format of the border around content boxes, can be: none, bars, coils, triangles, rectangle, rounded, roundedsmall, roundedright or faded
eyecatcher=false, %true, % Set to false for ignoring the left logo in the title and move the title left
headerheight=0.09\textheight, % Height of the header
headershape=roundedright, % Specify the rounded corner in the content box headers, can be: rectangle, small-rounded, roundedright, roundedleft or rounded
headerfont=\Large\bf\textsc, % Large, bold and sans serif font in the headers of content boxes
%textfont={\setlength{\parindent}{1.5em}}, % Uncomment for paragraph indentation
linewidth=2pt % Width of the border lines around content boxes
}
%----------------------------------------------------------------------------------------
% TITLE SECTION (spans whole width)
%----------------------------------------------------------------------------------------
%
{}
{\bf\textsc{Solving the Boltzmann equation for electron kinetics using Galerkin approach}\vspace{0.01em}} % Poster title
{\textsc{\small Milinda Fernando, Daniil Bochkov, Todd Oliver, Raja Laxminarayan, Philip Varghese, Robert Moser, and George Biros \hspace{12pt} \\The University of Texas at Austin}} % Author names and institution
{%\includegraphics[align=c,height=0.09\textheight]{pecos_roadmap_tst_1.5.pdf}% just figure (must edit fig to add highlight box)
\raisebox{-0.5\height}{\begin{overpic}[height=0.09\textheight]{pecos_roadmap_tst_1.5.pdf} % figure + highlight box
    \put(-1,8){\filldraw [semithick, fill=rightfooterorange, fill opacity=0.1, draw=gray, rounded corners] (1.9,1.2) rectangle +(1.15,0.7);}
\end{overpic}}%
\hspace{12pt} \includegraphics[align=c,height=3em]{psaap3-logo.png}%
\hspace{12pt} \includegraphics[align=c,height=3em]{oden_pecos_2020_wordmark.png}%
}

%----------------------------------------------------------------------------------------
%	OBJECTIVES
%----------------------------------------------------------------------------------------

\headerbox{Objectives}{name=objectives,column=0,row=0}{
\begin{itemize}
	\item \textbf{Key}: Develop scalable robust deterministic Boltzmann solver to determine species kinetics properties to enable accurate plasma simulations. 
	\item \textbf{Functional representation}: Identify robust, yet low cost efficient representation of the distribution function $f(t, \vect{v}, \vect{x})$.
	\item \textbf{Challenges}: Dimentionality of the problem (6 + 1 dimensions), numerical + HPC challenges.
\end{itemize}
\vspace{0.3em} % When there are two boxes, some whitespace may need to be added if the one on the right has more content
}

%----------------------------------------------------------------------------------------
%	INTRODUCTION
%----------------------------------------------------------------------------------------

\headerbox{Introduction}{name=introduction,column=1,row=0}{
\begin{itemize}
	\item Electron distribution function $f\of{t,\vect{v}, \vect{x}}$ defines the transport and kinetic properties.
	\item Evolution of $f$ is described by the Boltzmann equation.
	\begin{align}
		\partial_t f + \vect{v}\cdot \nabla_{\vect{x}} f  - \frac{\vect{E} q}{m} \cdot \nabla_{\vect{v }}f = C(f)
	\end{align}
	\item Challenges: 6+1 dimensions
	\item First, tackle the spatially homogeneous case, focusing on the representation of $f(\vect{v},t)$
	\begin{itemize}
		\item Use spherical harmonics for angular directions. 
		\item Radial direction approximations, with global and local approximations
	\end{itemize}
	\item \textbf{Goal}: Accurate representation of $f$, with minimum dofs. 
\end{itemize}
}

%----------------------------------------------------------------------------------------
%	RESULTS 1
%----------------------------------------------------------------------------------------

\headerbox{Results 1}{name=results,column=2,span=2,row=0}{

\begin{multicols}{2}
%\vspace{1em}
\begin{center}
\includegraphics[width=0.6\linewidth]{eye-candy/torch/coil-photo.png}
\captionof{figure}{Rogowski coils}
\end{center}
Aliquam auctor, metus id ultrices porta, risus enim cursus sapien,
quis iaculis sapien tortor sed odio. Mauris ante orci, euismod vitae
tincidunt eu, porta ut neque. Aenean sapien est, viverra vel lacinia
nec, venenatis eu nulla. Maecenas ut nunc nibh, et tempus
libero. Aenean vitae risus ante. Pellentesque condimentum dui. Etiam
sagittis purus non tellus tempor volutpat. Donec et dui non massa
tristique adipiscing.
\end{multicols}

%------------------------------------------------

\begin{multicols}{2}
\vspace{1em}
Sed fringilla tempus hendrerit. Vestibulum ante ipsum primis in faucibus orci luctus et ultrices posuere cubilia Curae; Etiam ut elit sit amet metus lobortis consequat sit amet in libero. Lorem ipsum dolor sit amet, consectetur adipiscing elit. Phasellus vel sem magna. Nunc at convallis urna. isus ante. Pellentesque condimentum dui. Etiam sagittis purus non tellus tempor volutpat. Donec et dui non massa tristique adipiscing. Quisque vestibulum eros eu.

\begin{center}
\includegraphics[width=0.6\linewidth]{eye-candy/torch/plume-laser.png}
\captionof{figure}{Cool laser}
\end{center}

\end{multicols}
}

%----------------------------------------------------------------------------------------
%	REFERENCES
%----------------------------------------------------------------------------------------

\headerbox{References}{name=references,column=0,above=bottom}{

\renewcommand{\section}[2]{\vskip 0.05em} % Get rid of the default "References" section title
\nocite{*} % Insert publications even if they are not cited in the poster
\small{ % Reduce the font size in this block
\bibliographystyle{unsrt}
\bibliography{sample} % Use sample.bib as the bibliography file
}}

%----------------------------------------------------------------------------------------
%	FUTURE RESEARCH
%----------------------------------------------------------------------------------------

\headerbox{Future Research}{name=futureresearch,column=1,span=3,aligned=references,above=bottom}{ % This block is as tall as the references block

\begin{multicols}{2}
Integer sed lectus vel mauris euismod suscipit. Praesent a est a est ultricies pellentesque. Donec tincidunt, nunc in feugiat varius, lectus lectus auctor lorem, egestas molestie risus erat ut nibh.

Maecenas viverra ligula a risus blandit vel tincidunt est adipiscing. Suspendisse mollis iaculis sem, in \emph{imperdiet} orci porta vitae. Quisque id dui sed ante sollicitudin sagittis.
\end{multicols}
}

%----------------------------------------------------------------------------------------
%	CONTACT INFORMATION
%----------------------------------------------------------------------------------------

%%% \headerbox{Contact Information}{name=contact,column=3,aligned=references,above=bottom}{ % This block is as tall as the references block
%%% \begin{description}\compresslist
%%% \item[Web] www.university.edu/smithlab
%%% \item[Email] john@smith.com
%%% \end{description}
%%% }

%----------------------------------------------------------------------------------------
%	CONCLUSION
%----------------------------------------------------------------------------------------

\headerbox{Conclusion}{name=conclusion,column=2,span=2,row=0,below=results,above=references}{

\begin{multicols}{2}

\tikzstyle{decision} = [diamond, draw, fill=blue!20, text width=4.5em, text badly centered, node distance=2cm, inner sep=0pt]
\tikzstyle{block} = [rectangle, draw, fill=blue!20, text width=5em, text centered, rounded corners, minimum height=4em]
\tikzstyle{line} = [draw, -latex']
\tikzstyle{cloud} = [draw, ellipse, fill=red!20, node distance=3cm, minimum height=2em]

\begin{tikzpicture}[node distance = 2cm, auto]
\node [block] (init) {Initialize Model};
\node [cloud, left of=init] (Start) {Start};
\node [cloud, right of=init] (Start2) {Start Two};
\node [block, below of=init] (init2) {Initialize Two};
\node [decision, below of=init2] (End) {End};
\path [line] (init) -- (init2);
\path [line] (init2) -- (End);
\path [line, dashed] (Start) -- (init);
\path [line, dashed] (Start2) -- (init);
\path [line, dashed] (Start2) |- (init2);
\end{tikzpicture}

%------------------------------------------------

\begin{itemize}\compresslist
\item Pellentesque eget orci eros. Fusce ultricies, tellus et pellentesque fringilla, ante massa luctus libero, quis tristique purus urna nec nibh. Phasellus fermentum rutrum elementum. Nam quis justo lectus.
\item Vestibulum sem ante, hendrerit a gravida ac, blandit quis magna.
\item Donec sem metus, facilisis at condimentum eget, vehicula ut massa. Morbi consequat, diam sed convallis tincidunt, arcu nunc.
\end{itemize}

\end{multicols}
}

%----------------------------------------------------------------------------------------
%	MATERIALS AND METHODS
%----------------------------------------------------------------------------------------

\headerbox{Materials \& Methods}{name=method,column=0,below=objectives,bottomaligned=conclusion}{ % This block's bottom aligns with the bottom of the conclusion block

The following materials were required to complete the research:

\begin{itemize}\compresslist
\item Curabitur pellentesque dignissim
\item Eu facilisis est tempus quis
\item Duis porta consequat lorem
\item Eu facilisis est tempus quis
\end{itemize}

The following equations were used for statistical analysis:

\begin{equation}
\cos^3 \theta =\frac{1}{4}\cos\theta+\frac{3}{4}\cos 3\theta
\label{eq:refname}
\end{equation}\

\begin{equation}
E = mc^{2}
\label{eqn:Einstein}
\end{equation}

Phasellus imperdiet, tortor vitae congue bibendum, felis enim sagittis lorem, et volutpat ante orci sagittis mi. Morbi rutrum laoreet semper. Morbi accumsan enim nec tortor consectetur non commodo nisi sollicitudin. Proin sollicitudin. Pellentesque eget orci eros. Fusce ultricies, tellus et pellentesque fringilla, ante massa luctus libero, quis tristique purus urna nec nibh.
}

%----------------------------------------------------------------------------------------
%	RESULTS 2
%----------------------------------------------------------------------------------------

\headerbox{Results 2}{name=results2,column=1,below=objectives,bottomaligned=conclusion}{ % This block's bottom aligns with the bottom of the conclusion block

Donec faucibus purus at tortor egestas eu fermentum dolor facilisis. Maecenas tempor dui eu neque fringilla rutrum. Mauris \emph{lobortis} nisl accumsan.

\begin{center}
\begin{tabular}{l l l}
\toprule
\textbf{Treatments} & \textbf{Response 1} & \textbf{Response 2}\\
\midrule
Treatment 1 & 0.0003262 & 0.562 \\
Treatment 2 & 0.0015681 & 0.910 \\
Treatment 3 & 0.0009271 & 0.296 \\
\bottomrule
\end{tabular}
\captionof{table}{Table caption}
\end{center}

Nulla ut porttitor enim. Suspendisse venenatis dui eget eros gravida tempor. Mauris feugiat elit et augue placerat ultrices. Morbi accumsan enim nec tortor consectetur non commodo.

\begin{center}
\begin{tabular}{l l l}
\toprule
\textbf{Treatments} & \textbf{Response 1} & \textbf{Response 2}\\
\midrule
Treatment 1 & 0.0003262 & 0.562 \\
Treatment 2 & 0.0015681 & 0.910 \\
Treatment 3 & 0.0009271 & 0.296 \\
\bottomrule
\end{tabular}
\captionof{table}{Table caption}
\end{center}
}

%----------------------------------------------------------------------------------------

\end{poster}

\end{document}
